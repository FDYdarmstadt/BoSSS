
%
% FDY TEMPLATE for ANUAL REPORT
% ===================================================================
\NeedsTeXFormat{LaTeX2e}
\documentclass[11pt,twoside,a4paper]{fdyartcl}
\usepackage[utf8]{inputenc}
\usepackage[T1]{fontenc}
\usepackage[ngerman,english]{babel} % selectlanguage wird nur dann gebraucht, wenn
                                    % mehrere Sprachen-packages verwendet werden,
                                    % das zuletzt angegebene package ist die aktive
                                    % Sprache, mit \selectlanguage kann umgeschaltet
                                    % werden
\usepackage[intoc]{nomencl} % zur Erstellung einer Nomenklatur
                                   % Die Option [intoc] sorgt dafuer,
                                   % dass die Nomenklatur im
                                   % Inhaltsverzeichnis eingetragen
                                   % wird. Aufruf mit:
                                   % makeindex Diss.nlo -s nomencl.ist -o Diss.nls
\usepackage{longtable}  % fuer tabellen, die evtl ueber Seiten
                        % umgebrochen werden muessen
\usepackage{graphicx}   % Stellt \includegraphics zur Verfuegung
\usepackage{parskip}    % Setzt parindent auf null und parskip auf
                        % einen angemessenen Wert
\usepackage{calc}       % Erlaubt, verschiedene Masse zu addieren
                        % z.B 1cm+2pt
\usepackage[a4paper,twoside,outer=2.2cm,inner=3cm,top=1.5cm,bottom=2.7cm,includehead]{geometry}
                                        % erheblich verbesserte
                                        % Papieranassung
\usepackage{setspace}   % Stellt \singlespacing, \onehalfspacing und
                        % \doublespacig zur Verfuegung
                        % erlaubt ausserdem die Verwendung der
                        % Umgebung \begin{spacing}{2.3}
\setstretch{1.05}       % minimal vergroesserter Zeilenabstand
\usepackage{amsmath}    % Stellt verschiedene Mathematik Operatoren
                        % und Befehle bereit und verbessert die
                        % Darstellung von Gleichungen ermoeglicht
                        % ausserdem die Verwendung von \boldsymbol
                        % fuer z.B. griechische Buchstaben
\usepackage{amsfonts}
\usepackage{amsthm}
\usepackage{mathpazo}   % Aenderung der Standardschrift auf Palatino
\usepackage{upgreek}    % nicht-kursive grichische Buchstaben
\usepackage{fancyhdr}

\graphicspath{{./Figures/}}%
% \input hyphen_dt.tex  % Spezielle Trennregeln fuer deutsche
                        % Woerter - gehoert in den Vorspann
\clubpenalty = 10000%
\widowpenalty = 10000%
\displaywidowpenalty =10000


%%%%%%%%%%%%%%%%%%%%%%%%%%%%%%%%%%%%%%%%%%%%%%%%%%%%%%%%%%%%%%%%%%%%%
%%%%%%%%%%%%%%%%%%%%%%%%%%%%%%%%%%%%%%%%%%%%%%%%%%%%%%%%%%%%%%%%%%%%%
% TITLE AND AUTHOR
%%%%%%%%%%%%%%%%%%%%%%%%%%%%%%%%%%%%%%%%%%%%%%%%%%%%%%%%%%%%%%%%%%%%%
%%%%%%%%%%%%%%%%%%%%%%%%%%%%%%%%%%%%%%%%%%%%%%%%%%%%%%%%%%%%%%%%%%%%%
\title{How to compile MUMPS for BoSSS\\TU Darmstadt - Chair of Fluid Dynamic (FDY)}
\author{Dennis Krause}
%%%%%%%%%%%%%%%%%%%%%%%%%%%%%%%%%%%%%%%%%%%%%%%%%%%%%%%%%%%%%%%%%%%%%
%%%%%%%%%%%%%%%%%%%%%%%%%%%%%%%%%%%%%%%%%%%%%%%%%%%%%%%%%%%%%%%%%%%%%
%%%%%%%%%%%%%%%%%%%%%%%%%%%%%%%%%%%%%%%%%%%%%%%%%%%%%%%%%%%%%%%%%%%%%

%%%%%%%%%%%%%%%%%%%%%%%%%%%%%%%%%%%%%%%%%%%%%%%%%%%%%%%%%%%%%%%%%%%%%
%%%%%%%%%%%%%%%%%%%%%%%%%%%%%%%%%%%%%%%%%%%%%%%%%%%%%%%%%%%%%%%%%%%%%
% Kopfzeile
\pagestyle{fancy}
\fancyhf{}
\fancyhead[EL]{\sffamily \small  \thepage\ }
\fancyhead[OR]{\sffamily \small  \thepage\ }
\fancyhead[OC]{\sffamily \small TU Darmstadt - FDY}
\fancyhead[EC]{\sffamily \small BoSSS tutorial}
%%%%%%%%%%%%%%%%%%%%%%%%%%%%%%%%%%%%%%%%%%%%%%%%%%%%%%%%%%%%%%%%%%%%%
%%%%%%%%%%%%%%%%%%%%%%%%%%%%%%%%%%%%%%%%%%%%%%%%%%%%%%%%%%%%%%%%%%%%%
%%%%%%%%%%%%%%%%%%%%%%%%%%%%%%%%%%%%%%%%%%%%%%%%%%%%%%%%%%%%%%%%%%%%%
%\renewcommand{\vec}[1]{\textrm{\textbf{#1}}}
\renewcommand{\vec}[1]{\mathbf{#1}}
\newcommand{\grvec}[1]{\boldsymbol{#1}}
\renewcommand{\div}[1]{\textrm{div}\left( #1 \right)}

\newcommand{\real}{\mathbb{R}}
\newcommand{\complex}{\mathbb{C}}
\newcommand{\realpos}{\mathbb{R}_{\geq0}}
\newcommand{\natWoZero}{\mathbb{N}_{>0}}
\newcommand{\natInclZero}{\mathbb{N}_{\geq0}}
\newcommand{\nat}{\mathbb{N}}
\newcommand{\integers}{\mathbb{Z}}
\newcommand{\code}[1]{\sffamily{#1}}
\newcommand{\coderm}[1]{\footnote{\code{#1}}}


\newcounter{cnt}

\newtheoremstyle{myPlain} % namei
 {15pt}% Space above
 {3pt}% Space below
 {\itshape}% Body font
 {}% Indent amount
 {\bfseries}% Theorem head font %\scshape
 {:}% Punctuation after theorem head
 {.5em}% hSpace after theorem headi2
 {\thmname{#1} \thmnumber{#2} \thmnote{ -- #3}}% Theorem head spec (can be left empty, meaning `normal')


\theoremstyle{myPlain}
%\theoremstyle{plain}
\newtheorem{myTheorem}[cnt]{Theorem}
%\newtheorem{satzDef}[cnt]{Satz und Definition}

\newtheoremstyle{myDefinition} % namei
 {15pt}% Space above
 {3pt}% Space below
 {}% Body font
 {}% Indent amount
 {\bfseries}% Theorem head font
 {:}% Punctuation after theorem head
 {.5em}% hSpace after theorem headi2
 {\thmname{#1} \thmnumber{#2} \thmnote{ -- #3}}% Theorem head spec (can be left empty, meaning `normal')

\theoremstyle{myDefinition}
%\theoremstyle{definition}
%\newtheorem{bsp}[cnt]{Beispiel}
\newtheorem{myDef}[cnt]{Definition}
\newtheorem{myRem}[cnt]{Remark}
\newtheorem{myNot}[cnt]{Notation}


%%%%%%%%%%%%%%%%%%%%%%%%%%%%%%%%%%%%%%%%%%%%%%%%%%%%%%%%%%%%%%%%%%%%%

%       DOKUMENT
\begin{document}

\pagenumbering{arabic} % Zurueckschalten auf arabische Ziffern, dabei wird der Zaehler auf 1 gesetzt


% Titelseite
% -------------------
\maketitle


\begin{abstract}
This document explains the compile the direct solver MUMPS for parallel Usage on Windows and on the Lichtenberg cluster. The method described holds for MUMPS\_5.0.2.
\end{abstract}

\section{Lichtenberg}
In the following the compilation of MUMPS on the Lichtenberg cluster will be explained.
\begin{enumerate}
	\item Download MUMPS\_5.0.2 from the MUMPS website and unzip it.
	\item Copy the \textbf{Makefile.inc} into the main MUMPS folder and the \textbf{build\_shared} script to the \textit{/lib} directory. Both files you can find in the underlying folder \textit{Lichtenberg}.
	\item Load modules \textbf{intel/2016u4} (2017 does not work) and one \textbf{openmpi/intel/2.0.1}.
	\item Execute the makefile by typing 'make'.
	\item If all went well there should be three libraries, naming libdmumps.a, libmumps\_common.a and libpord.a in your \textit{/lib} folder.
	\item Execute the \textbf{build\_shared} by typing './build\_shared'.
	\item Modify your \textbf{.bashrc} and add the path to your libdmumps.so.	
\end{enumerate}
Now it should be possible to run the MUMPS solver on the LichtenbergCluster in the same way you do on windows. 

The performance of MUMPS can possibly be increased by linking against Metis and Parmetis. At this stage there was a conflict between BoSSS, using Parmetis 3 and MUMPS, using Parmetis 4. As a reason the software Pord, which is distributed with MUMPS is being used for partitioning.

\section{Windows}
Usually the MUMP-library is distributed with installing BoSSS on your local workstation. Therefore you do not have to bother about compiling MUMPS under Windows. If you need to, for any reason I feel sorry for, compile MUMPS again on Windows the following steps show the possible workflow which worked out for us:
\begin{enumerate}
	\item Use MinGW to compile MUMPS with the \textbf{Makefile.inc} provided to yield get *.lib files in your \textit{/lib} directory.
	\item Use a visual studio project 2008 to build a shared library. An example can be found under \textit{.../scratch/krause/MUMPS}.
	\begin{enumerate}
		\item Be careful to link all the libraries required:
		\begin{itemize}
			\item libdmumps.lib 
			\item libmumps\_common.lib 
			\item libpord.lib
			\item msmpi.lib 
			\item msmpifec.lib 
			\item mkl\_scalapack\_lp64.lib 
			\item mkl\_blacs\_msmpi\_lp64.lib 
			\item mkl\_intel\_thread.lib 
			\item mkl\_core.lib 
			\item libiomp5md.lib 
			\item mkl\_intel\_lp64.lib
		\end{itemize}
		\item Use a \textbf{*.def} file to define the possible entry points for your dynamic library. Functions you have to define are:
		\begin{itemize}
			\item	dmumps\_f77\_=DMUMPS\_F77 (very important to redefine the name of this function, otherwise MUMPS cannot find the entry point)
			\item 	mumps\_get\_mapping
			\item 	mumps\_get\_pivnul\_list
			\item	mumps\_get\_sym\_perm
			\item mumps\_get\_uns\_perm
		\end{itemize}	
	\end{enumerate}
\end{enumerate}
\section{Possible Errors using MUMPS}
MUMPS is very sensitive in terms of the estimated memory needed in comparison with the real memory consumption. If you yield an Error or a high residual from solver you should comment the No Output parameters in \textbf{MUMPSSolver.cs} and run the code again. 

After that you can analyse your error by using the error codes given by MUMPS and the MUMPS userguide and can fit your control parameters to your specific needs.

 \textbf{ATTENTION:} The control parameter array ICNTL in the userguide starts with 1 and in the c\#-code it starts with 0. Therefore ICNTL[0] in the Code is ICNTL[1] in the handbook and so on. 



\end{document} 