\section{The discretization of spatial differential operators}

\begin{myDef}[Conservative Operator]
For...
\begin{packed_itemize}
  \item a list of \emph{domain variables}, $U := (u_0,\cdots,u_{\Lambda-1}) \in \ceins(\Omega)^\Lambda$ and
  \item a list of \emph{codomain variables}, $W := (w_0,\cdots,w_{\Gamma-1}) \in L^2(\Omega)^\Gamma$
\end{packed_itemize}
a mapping
\[
Op:\ceins(\Omega)^\Lambda \rightarrow L^2(\Omega)^\Gamma, \ U \mapsto W = Op(U)
\]
is called \emph{conservative operator} if
every codomain variable $w_\gamma$ can be written as
\[
  w_\gamma = \divergence{ \vec{f}_\gamma(\vec{x},U) } + q_\gamma(\vec{x},U).
\]
The functions
\begin{packed_itemize}
 \item $ \vec{f}_\gamma(\vec{x},U) \in \left( \ceins (\real^D \times \real^\Lambda) \right)^D$
      are called the flux and
 \item $  q_\gamma(\vec{x},U) \in L^2(\real^D \times \real^\Lambda)$
      are called the source
\end{packed_itemize}
of codomain variable $w_\gamma$.
\label{ConservativeOp}
\end{myDef}

\begin{myNot}[Outer normal]
For $\vec{x} \in \partial K_j$, $\vec{n}_\vec{x} \in \partial \mathcal{S}_D$ denotes an outer normal
field for $K_{\tau(j)}$, for
those points of $\partial K$ where an outer normal is defined, which may not be the case
for a null-subset of $\partial K$
i.e. ``edges'' and ``corners''.
\end{myNot}

\begin{myDef}[Conservative operators on product spaces of $DG_p$]
Again, like in (\ref{ConservativeOp}), there is
a list of domain variables,
$U := (u_0,\cdots,u_{\Lambda-1}) \in DG_{(p_0, \cdots, p_{\Lambda-1})} =: Dom$
and a list of codomain variables
$W := (w_0,\cdots,w_{\Gamma-1}) \in DG_{(l_0, \cdots, l_{\Gamma-1})} =: Cod$.
A mapping
\[
 OP_h :
 DG_{p_0} \times \cdots \times DG_{p_{\Lambda-1}}
 \rightarrow
 DG_{l_0} \times \cdots \times DG_{l_{\Gamma-1}},
 \
 U \mapsto W = Op_h(U)
\]
is called a DG-discretization of the conservative operator $Op$
if the $n$-th DG coordinate of variable $w_\gamma$ in cell $K_{\tau(j)}$, $\tilde{w}_{\gamma,j,n}$
can be written as
\[
  \tilde{w}_{\gamma,j,n} =
  \int_{\partial K_{\tau(j)}}
       F_\delta \ \phi_{j,n}
  \ \textrm{dS}
  -
  \int_{K_{\tau(j)}} \left(
       \vec{f}_\gamma \cdot \nabla \phi_{j,n}
       -
       q_i \ \Phi_{j,n}
  \right) \ \textrm{d}\vec{x}
  .
\]
The function $F_\delta(\vec{x},U^\textrm{in},U^\textrm{out},\vec{n}_\vec{x})$ is called the \emph{Riemannian}
for the flux $\vec{f}_\gamma(\vec{x},U)$. Their properties are described in below, see (\ref{riemannians}).
Here, for $\vec{x} \in \partial K_{\tau(j)}$
\[
  U^\textrm{in} := \lim_{\stackrel{\vec{y} \rightarrow \vec{x}}{\vec{y} \in K_{\tau(j)}}}(U(\vec{y}))
  \textrm{ and }
  U^\textrm{out} := \lim_{\stackrel{\vec{y} \rightarrow \vec{x}}{\vec{y} \notin K_{\tau(j)}}}(U(\vec{y}))
  .
\]
By coordinate mappings, the conservative operator can be written as a mapping
\[
  Op_h: \real^{dim(Dom)} \rightarrow \real^{dim(Cod)}, \quad \tilde{U} \mapsto \tilde{W}.
\]
\label{ConservativeOp_h}
\end{myDef}

\begin{myRem}[On Riemannians]
The following properties are essential for proper defined Riemannians:
\begin{packed_itemize}
  \item $F_\gamma(\vec{x},U,U,\vec{n}_{\vec{x}}) = \vec{n}_{\vec{x}} \cdot \vec{f}_\gamma(\vec{x},U)$,
  i.e ``$F_\gamma$ is an approximation to $\vec{n}_{\vec{x}} \cdot \vec{f}_\gamma$''
  \item $F_\gamma(\vec{x},U,V,\vec{n}_{\vec{x}}) = -F_\gamma(\vec{x},V,U,-\vec{n}_{\vec{x}})$
  \item $\left| F_\gamma(\vec{x},U,V,\vec{n}_{\vec{x}}) - \vec{n}_{\vec{x}} \cdot \vec{f}_\gamma(\vec{x},U) \right| \leq L | U-V |$
  for some Lipschitz constant $L \in \realpos$
\end{packed_itemize}
\label{riemannians}
\end{myRem}

\begin{myDef}[Consistency of operators]
An operator $Op_h$ like in (\ref{ConservativeOp_h}) is called consistent
with an operator $Op$ like in (\ref{ConservativeOp}) with convergence
order $k$ if
\[
 \left\| Op_h(Proj_p(U)) - Proj_p( Op(U) ) \right\|_2 \leq h_\textrm{max}^k \cdot c(U)
\]
for any $U \in \ceins(\Omega)^\Lambda$ and a constant $c(U) \in \realpos$ which depends on $U$.
\label{consistency_of_operators}
\end{myDef}

\begin{myRem}
The motivation for (\ref{consistency_of_operators}) is the
following ``approximately'' commutative diagram:
%\begin{center}
\[
\begin{xy}
  \xymatrix{
      H^1(\Omega)^\Lambda \ar[r]^{Op} \ar[d]_{Proj}  &   (L^2(\Omega))^\Gamma  \ar[d]^{Proj}  \\
      \prod_\delta DG_{p_\delta} \ar[r]_{Op_h}       &  \prod_\gamma DG_{l_\gamma}
  }
\end{xy}
% ----------------------------------------#
\quad
\begin{xy}
  \xymatrix{
      U \ar@{|->}[rrrr]^{Op} \ar@{|->}[d]_{Proj} & &             &          &  W  \ar@{|->}[d]^{Proj}  \\
      U_h \ar@{|->}[rr]_{Op_h}                   & &  Op_h(U_h)  &  \approx &  W_h
  }
\end{xy}
\]
(Here, $Proj$ is an abbreviation for
$( Proj_{p_0}, \cdots, Proj_{p_{\Lambda-1}} )$
and
$( Proj_{l_0}, \cdots, Proj_{l_{\Gamma-1}} )$.
\end{myRem}

\begin{myRem}[Additive component decomposition of conservative operators in BoSSS]
In BoSSS, a conservative operator is called
\emph{spatial difference operator}\coderm{BoSSS.Foundation.SpatialDifferenceOperator}
and is specified as an additive composition of operators, which are called
\emph{equation components}\coderm{BoSSS.Foundation.IEquationComponent}.
The domain and codomain variables for the  spatial difference operator, are specified as lists
of symbolic names\coderm{BoSSS.Foundation.SpatialDifferentialOperator.DomainVar} \coderm{BoSSS.Foundation.SpatialDifferentialOperator.CodomainVar}.
Each equation component itself is a conservative operator with exactly one codomain variable,
i.e. it maps $\prod_\delta DG_{p_\gamma} \rightarrow DG_l$.
The domain variables of the equation components are specified as some
sub-list\coderm{BoSSS.Foundation.IEquationComponent.ArgumentOrdering}
of the domain variables list of the spatial difference operator.
The equation components may be classified into \emph{fluxes} and \emph{sources},
and wether they are either \emph{linear} or \emph{nonlinear}:
\begin{packed_itemize}
  \item nonlinear fluxes\coderm{BoSSS.Foundation.INonlinearFlux},\coderm{BoSSS.Foundation.INonlinearFluxEx};
  \item nonlinear sources\coderm{BoSSS.Foundation.INonlinearSource};
  \item linear fluxes\coderm{BoSSS.Foundation.ILinearFlux};
  \item linear sources\coderm{BoSSS.Foundation.ILinearSource};
\end{packed_itemize}
If more than one equation component is specified for one codomain variable, they are
added\coderm{BoSSS.Foundation.SpatialDifferentialOperator.EquationComponents}.
\end{myRem}

\begin{myRem}[Evaluation of conservative operators in BoSSS]
A conservative operator, or spatial differential operator $Op_h$ can be evaluated
in BoSSS
by\coderm{BoSSS.Foundation.SpatialDifferentialOperator.Evaluate(...)}
or\coderm{BoSSS.Foundation.SpatialDifferentialOperator.Evaluator.Evaluate(...)}.
If $Op_h$ is affine-linear, i.e.
\[
  Op_h(\tilde{U}) = \mathcal{M} \cdot \tilde{U} + b,
\]
the (sparse) matrix $\mathcal{M}$ and the affine vector $b$ can
be computed
by\coderm{BoSSS.Foundation.SpatialDifferentialOperator.ComputeMatrix(...)}.
\end{myRem}


%\subsection{Nonlinear fluxes}

%\subsection{Nonlinear sources}


%\subsection{Linear sources}

%\subsection{Fluxes for Interior Penalty Methods}
