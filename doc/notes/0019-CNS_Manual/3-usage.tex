\section{Controlling \emph{CNS}}
\label{sec:usage}

All \emph{BoSSS} applications are controlled by means of an XML control file.
In the following, it will be assumed that the reader is familiar with the basic
structure of this file. As a result, only the most important points specific to
\emph{CNS} will be discussed in detail.

First of all, the polynomial degrees for the density (denoted by \texttt{rho}),
the momentum field (denoted by \texttt{m}) and the energy (denoted by
\texttt{rhoE}) have to be specified (see listing \ref{lst:degrees}, lines 3 to 5).
Since the momentum is a vector-valued quantity, it will consist of
one, two or three components depending on the spatial dimension of the considered
problem domain (i.e. of the grid). The components of this vector have the same
polynomial degree and can be addressed via \texttt{m0} to \texttt{m2} in other
parts of the control file.

\begin{lstlisting}[caption={Specification of the polynomial degrees}, 
label={lst:degrees}]
<fields_degree>
	<field identification="rho" degree="1"/>
	<field identification="m" degree="2"/>
	<field identification="rhoE" degree="3"/>
	
	<!-- Optional auxiliary variables -->
	<field identification="p" degree="1"/>
	<field identification="u" degree="1"/>
	<field identification="e" degree="1"/>
	<field identification="T" degree="1"/>
</fields_degree>
\end{lstlisting}

In addition to these required specifications, \emph{CNS} supports the
automatic calculation of auxiliary variables (see listing \ref{lst:degrees},
lines 8 to 11). The values of auxiliary variables specified in the
\texttt{fields\_degree} section of the control file are calculated in every
time-step (using a basis of the specified degree) and are included in all IO
operations (i.e. they will be included in plots and/or stored in the database).
Currently supported variables are the pressure (denoted by \texttt{p}), the
velocity field (denoted by \texttt{u}), the specific inner energy (denoted by
\texttt{e}) and the absolute temperature (denoted by \texttt{T}). Please note
that, just as in the case of the momentum field, the velocity field is a
vector-valued quantity. That is, it consists of one, two or theree components
which can be addressed via \texttt{u0} to \texttt{u2} in other parts of the
control file.

Auxiliary variables are also supported when specifying initial conditions. In
listing \ref{lst:initial_conditions}, for example, initial values have been
specified for the density, the components of the momentum field and the pressure
\texttt{p}. \emph{CNS} will use this information to calculate the initial energy
automatically. Using an auxiliary variable in the initial conditions is,
however, independent of using an auxiliary variable in the list of polynomial
degrees. In the given example, this means that the pressure will not be an IO
variable by default. It is important to note that this calculation may have
influences on the quality of the projection onto the Discontinuous Galerkin
space if the projected function cannot be represented exactly by the chosen
polynomial space.


\begin{lstlisting}[caption={Specification of initial conditions}, 
label={lst:initial_conditions},float]
<initial mode="values">
	<values>
		<formula>rho(x,y,z) = 1</formula>
		<formula>m0(x,y,z) = y</formula>
		<formula>m1(x,y,z) = z</formula>
		<formula>m2(x,y,z) = x</formula>
		<formula>p(x,y,z) = x^2 + y^2 + z^2</formula>
	</values>
</initial>
\end{lstlisting}

In general, all auxiliary variables mentioned above are also supported in the initial
conditions sections. It should be obvious, however, that not all combinations of
initial values lead to a well-defined problem. This fact can be illustrated by
considering a case where initial values for the components of the momentum field,
the inner energy and the temperature are specified in the control file. Since the inner
energy and the temperature of an ideal gas are equivalent, \emph{CNS} will report
an error in such a situation.

Finally, \emph{CNS} defines some additional parameters in the \texttt{properties}
section of the control file (see listing \ref{lst:properties}). Table
\ref{fig:properties} summarizes the available options and the respective admissible 
settings.

\begin{lstlisting}[caption={Specification of \emph{CNS}-specific properties}, 
label={lst:properties},float]
<properties>
	<string key="equationSystem">Euler</string>
	<string key="convectiveFluxType">HLLC</string>
	<string key="diffusiveFluxType">dGRP</string>
	<float key="kappa">1.4</float>
	<float key="Reynolds">10.0</float>
	<float key="Mach">0.1</float>
	<float key="Prandtl">0.71</float>
	<string key="timeStepping">explicit</string>
	<string key="explicitScheme">explicitEuler</string>
	<string key="implicitScheme">implicitEuler</string>
	<string key="implicitSolver">monkey</string>
</properties>
\end{lstlisting}

\begin{table}
	\centering

	\begin{tabular}{l | p{8cm} | p{3cm}}
		\hline
		Option & Description & Admissible \hspace{3cm} values\\
		
		\hline\hline
		Equation system
		& Defines the system of partial differential equations to be solved
		& \texttt{Euler} \hspace{3cm} \texttt{Stokes}* \hspace{3cm} 
		\texttt{Navier-Stokes}* \\
		
		\hline
		Convective flux type
		& Chooses the flux function for the convective part of the equation system.
		Irrelevant if the equation system is set to \texttt{Stokes}
		& \texttt{HLL} \hspace{3cm} \texttt{HLLC}\\
		
		\hline
		Diffusive flux type
		& Chooses the flux function for the diffusive part of the equation system.
		Irrelevant if the equation system is set to \texttt{Euler}
		& \texttt{dGRP}*\\
		
		\hline
		Kappa
		& The heat capacity ratio $\kappa$
		& Positive floats\\
		
		\hline
		Reynolds
		& The reference Reynolds number. Irrelevant if the equation system is set to 
		\texttt{Euler}
		& Positive floats\\
		
		\hline
		Mach
		& The reference Mach number
		& Positive floats\\
		
		\hline
		Prandtl
		& The reference Prandtl number. Irrelevant if the equation system is set to 
		\texttt{Euler}
		& Positive floats\\
		
		\hline
		Time stepping
		& Defines the general structure of the time-stepping scheme for the
		convective and the diffusive part of the configured equation system. The
		option \texttt{mixed} treats the convective part explicitly and the
		diffusive part implicitly and is thus illegal if the equation system is
		set to \texttt{Euler}
		& \texttt{explicit} \hspace{3cm} \texttt{mixed}\\
		
		\hline
		Explicit scheme
		& The explicit time-stepping scheme to be used
		& \texttt{explicitEuler} \hspace{3cm} \texttt{rungeKutta} \hspace{3cm}
		\texttt{heun}\\
		
		\hline
		Implicit scheme
		& The implicit scheme to be used. Only relevant if the time stepping is set
		to \texttt{mixed}
		& \texttt{implicitEuler} \hspace{3cm} \texttt{crankNicolson}\\
		
		\hline
		Implicit solver
		& The solver used for the solution of the system of equations emerging
		from the implicit treatment of the diffusive part. Irrelevant if the
		time stepping is set to \texttt{explicit}
		& Any defined linear solver\\
		
		\hline
	\end{tabular}
	
	\caption{Summary of \emph{CNS}-specific control options. Admissible values marked
	with an asterisk are still under development}
	\label{fig:properties}
\end{table}
