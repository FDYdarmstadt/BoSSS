\section{Numerical examples}
\label{sec:examples}

This section briefly outlines two numerical experiments and the respective
results that have been used to validate \emph{CNS}. It should however be noted
that the given examples only cover a subset of all possible applications and
that a more thorough validation has to be a subject of further studies.
Nevertheless, the obtained results are encouraging and suggest that the overall
implementation produces reasonable results. Further studies will focus on
the role of the boundary conditions in different flow configurations and the
convergence properties of the implemented scheme.


\subsection{Shock tube problems}

Shock tube problems have been studied extensively in the context of approximate
Riemann solvers (e.g., see \cite{Toro2009}) because they are realizations of the
Riemann problem \ref{eqn:Riemann_problem}. Hence, they are an adequate measure
for the validation of the implementation of the numerical flux function
introduced in section \ref{sec:numerics}.

Figure \ref{fig:shock_tube_initial}
shows the initial configuration of such a shock tube problem. The diaphragm in
the middle separates two regions (denoted by 1 and 4) with different densities
and pressures. At $t=0$, the diaphragm breaks and a set of waves starts to
spread from the plane of discontinuity. A qualitative outline of this pattern is
given in figure \ref{fig:shock_tube_final}. In regions 1 and 4, the fluid is
still unaffected by the disturbances due to the finite speed of wave
propagation. On the other hand, two new regions (denoted by 2 and 3) have
formed. While the transition between region 1 and region 2 is given by a
right-travelling shock wave, regions 2 and 3 are separated by a 
contact wave (i.e. the pressure is continuous as discussed in section
\ref{sec:introduction}) following the shock. On the other side of the tube,
regions 3 and 4 are separated by smooth, left-travelling expansion wave.

\begin{figure}
	\centering
	\includegraphics[width=0.8\textwidth]{shockTubeDescriptionInitial}
	\caption{Initial configuration of a shock tube problem. Regions 1 and
	4 are separated by a diaphragm that breaks at $t=0$}
	\label{fig:shock_tube_initial}
\end{figure}

\begin{figure}
	\centering
	\includegraphics[width=0.8\textwidth]{shockTubeDescriptionFinal}
	\caption{Configuration in the shock tube after the diaphragm has broken
	($t>0$)}
	\label{fig:shock_tube_final}
\end{figure}

The classical configuration often referred to as \emph{Sod shock tube} with
initial conditions $\vec{u} = \vec{0}$, $\rho_1 = 0.125$, $p_1 = 0.1$,
$\rho_4 = 1.0$ and $p_4 = 1.0$ has been simulated in one, two and three space
dimensions. The rectilinear grid invariably consisted of 100 cells in x
direction, while the y and the z direction were, where applicable, discretized
each with 10 cells. Please note that this test case could only be conducted
using zeroth order polynomials because the discontinuous nature of the exact
solution requires the application of limiting techniques in order to avoid
non-physical oscillations (also known as \emph{Gibbs phenomenon}) if higher
order polynomials are employed.

\begin{figure}
	\centering
	\subfigure[Pressure]{
		\includegraphics[width=.45\textwidth]{shockTube2d2dsPressure}
		\label{fig:shockTubePressure}}
	\quad
	\subfigure[Mach number]{
		\includegraphics[width=.45\textwidth]{shockTube2d2dsMach}
		\label{fig:shockTube1dMach}}
	\caption{State inside the shock tube at $t=0.2$}
	\label{fig:shockTubeResults}
\end{figure}

As expected, the numerical results were virtually identical in all three cases
which is why only the one-dimensional results will be discussed in the
following. In figure \ref{fig:shockTubeResults}, the numerical solution obtained
is compared to the exact solution which can e.g. be found in
\cite{Anderson2002}. The numerical solution obviously is in good agreement
with the analytical results which implies a correct implementation of the HLLC
flux presented in section \ref{sec:numerics}. However, the given test case is
hardly influenced by boundary conditions which is why this aspect will be the
focus of the following test cases.


\subsection{Subsonic flow through a converging-diverging nozzle}

As a second example, we investigate the subsonic flow through a symmetric
converging-diverging nozzle. The upper half of the problem geometry (which goes
back to \cite{Liou1987}) is displayed in figure \ref{fig:nozzle_grid}. Its cross
sectional area $A$ is parametrized according to
\begin{equation}
	\label{eqn:parametrizationNozzle}
	A = \begin{cases}
		1.75 - 0.75 \cos((0.2 x - 1.0) \pi)
			& \mathrm{if~~} x < 5.0\\
		1.25 - 0.25 \cos((0.2 x - 1.0) \pi)
			& \mathrm{if~~} 5.0 \leq x \leq 10.0
	\end{cases}
\end{equation}
which leads to a throat area of $1.0$. Due to the symmetry of the problem, only
the upper part of the geometry is meshed while the centreline is treated as a
symmetry plane. For the following considerations, a coarse, a medium and a fine
grid consisting of 1520, 6376 and 10015 cells respectively have been used.

\begin{figure}
	\centering
	\includegraphics[width=\textwidth]{nozzle81CoarseGrid}
	\caption{Computational domain and coarse grid for the flow through a
	converging-diverging nozzle}
	\label{fig:nozzle_grid}
\end{figure}

\cite{Anderson2002} gives an analytical solution for the flow conditions at the
centreline for a quasi one-dimensional flow starting from reservoir conditions
(i.e. the total pressure $p_t$ and the total temperature $T_t$ of the fluid at
rest are given at the inlet) with a prescribed pressure $p_\mathrm{exit}$ at the
outlet. In this particular case, the settings $p_t = 1.0$, $T_t = 1.0$ and
$p_\mathrm{exit} = 0.81$ have been used. This leads to subsonic flow conditions
throughout the whole nozzle and a Mach number of
$\mathrm{Ma}_\mathrm{throat} \approx 0.87$ in centre of the throat. Moreover,
the throat area exhibits strong gradients of the flow parameters in this
configuration which renders it an ideal candidate for the evaluation of the
quality of numerical solutions.

\begin{figure}
	\centering
	\subfigure[Pressure]{
		\includegraphics[width=.45\textwidth]{nozzle81comparisonPressure}
		\label{fig:nozzle81comparisonPressure}}
	\quad
	\subfigure[Mach number]{
		\includegraphics[width=.45\textwidth]{nozzle81comparisonMach}
		\label{fig:nozzle81comparisonMach}}
	\caption{Comparison of the results at the centreline of
		the converging-diverging}
	\label{fig:nozzle81comparison}
\end{figure}

An illustration of the results obtained using \emph{CNS} can be found in figure
\ref{fig:nozzle81comparison} where the computed pressures and Mach numbers
are compared to the respective theoretical values. The zeroth order solution on
the coarse grid significantly underestimates the Mach number and overestimates
the pressure in the throat. As expected, the agreement with the exact solution
can be improved significantly by either increasing the number cells or the
polynomial degree.

It should however be pointed out that the use of first order Ansatz functions
leads to a considerably better agreement with the analytical solution compared
to the zeroth order solution on grids with a higher overall number of degrees
of freedom. This result is encouraging and demonstrates the potential advantage
over classical Finite Volume schemes.
