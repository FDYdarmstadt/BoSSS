\NeedsTeXFormat{LaTeX2e}
\documentclass[11pt,twoside,a4paper]{fdyartcl}
\usepackage[utf8]{inputenc}
\usepackage[T1]{fontenc}
\usepackage[ngerman,english]{babel}
\usepackage{longtable}
\usepackage{graphicx}
\usepackage{parskip}
\usepackage{calc}
\usepackage[a4paper,twoside,outer=2.2cm,inner=3cm,top=1.5cm,bottom=2.7cm,includehead]{geometry}
\usepackage{setspace}
\setstretch{1.05}
\usepackage{amsmath}
\usepackage{amsfonts}
\usepackage{amsthm}
\usepackage{mathpazo}
\usepackage{upgreek}
\usepackage{fancyhdr}
\usepackage{ngerman}
\usepackage{todo}
\usepackage{hyperref}
\usepackage{listings}

\graphicspath{{./Figures/}}
\clubpenalty = 10000%
\widowpenalty = 10000%
\displaywidowpenalty =10000
\bibliographystyle{plain}

\lstset{language=[Sharp]C, basicstyle=\footnotesize, captionpos=b,%
  frame=single, numbers=left, numberstyle=\footnotesize, numbersep=5pt,%
  breaklines=true, float}

\title{LDG Tutorial}
\author{Bj"orn M"uller}

\pagestyle{fancy}
\fancyhf{}
\renewcommand{\headrulewidth}{0pt}

\fancyhead[CE]{\sffamily \small \thepage \quad \hrulefill \quad \sffamily
\small BoSSS documentation } \fancyhead[CO]{\sffamily \small B. M"uller \quad
\hrulefill \quad \sffamily \small \thepage }

% Dieser Befehl stellt sicher, dass neue Kapitel auf rechten (ungeraden)
% Seiten beginnen
\newcommand{\clearemptydoublepage}%
{\newpage{\pagestyle{empty}\cleardoublepage}}
\newfont{\myrm}{cmr12 at 12 pt}

% FigureXYLabel - urspruenglich in defin.tex, von Prof.~Dr.-Ing.~M. Oberlack
% Urspruengliche Version umfasste 5 Parameter - geaenderte 7 Parameter -
% Bauerbach
% 1 Figurename: \includegraphics[......
% 2 Beschriftung der x Achse
% 3 x-Beschriftung - Verruecken horizontal positiv nach links
% 4 x Beschriftung - Verruecken vertikal positiv nach unten
% 5 Beschriftung der y Achse
% 6 y-Beschriftung - Verruecken horizontal positiv nach links
% 7 y Beschriftung - Verruecken vertikal positiv nach oben
\newlength{\FigureHeight}
\newlength{\FigureHeightHalf}
\newcommand{\FigureXYLabel}[7]{%
\settoheight{\FigureHeight}{#1}%
\setlength{\FigureHeightHalf}{0.5\FigureHeight}%
\addtolength{\FigureHeightHalf}{#7}%
\raisebox{\FigureHeightHalf}{\makebox[0cm][r]{#5\makebox[#6]{}}}%
#1\\%
\vspace{#4}%
{\makebox{#2\makebox[#3]{}}}}

% Scientific notation for numbers
\providecommand{\e}[1]{\ensuremath{\times 10^{#1}}}


\begin{document}
\pagenumbering{arabic}
\maketitle

\begin{abstract}
The present article illustrates the implementation of LDG schemes in BoSSS. It
does so by introducing three simple examples and explaining a possible 
realization in BoSSS.
\end{abstract}

\section{Theoretical background}
\label{sec:theoreticalBackground}

The LDG method as introduced by \cite{Cockburn1998} can be used in order to 
solve scalar second order conservation laws of the form
\begin{equation}
	\frac{\partial u}{\partial t}
	+ \frac{\partial}{\partial x_i} \left(
	  	f_i(u)
	  	- a_{ij}(u) \frac{\partial u}{\partial x_j}
	\right)
	= Q ~~~ i,j = 1,\ldots,d
\label{eq:generalForm}
\end{equation}
where $u$ is the unknown scalar quantity,
$f = (f_1, \ldots, f_d): \mathbb{R} \rightarrow \mathbb{R}^d$
is a vector of arbitrary functions,
$a_{ij}: \mathbb{R} \rightarrow \mathbb{R}^{d,d}$ is a positive semidefinite
matrix, $Q$ is a source term and $d$ is the number of spatial dimensions. Since
$a_{ij}$ is positive semidefinite, there exists a symmetric matrix $b_{ij}$
such that the relation
\begin{equation}
	a_{ij} = b_{ik} b_{kj} ~~~ k = 1,\ldots,d
\label{eq:decomposition}
\end{equation}
holds.

When introducing \ref{eq:generalForm} into the Discontinuous Galerkin
framework, one has to find a way of dealing with the second order spatial
derivatives. One method in this context is the so called LDG method which
simply rewrites the considered equation as a system of first order equations
by introducing auxiliary variables $q = (q_1, \ldots, q_d)^T$. That way,
(\ref{eq:generalForm}) corresponds to
\begin{align}
\label{eq:generalFirstOrderSystem_u}
	\frac{\partial u}{\partial t}
	+ \frac{\partial}{\partial x_i} \left(
		f_i(u)
		- b_{ij}(u) q_j
  \right)
  &= Q\\
\label{eq:generalFirstOrderSystem_q}
  \frac{\partial g_{jk}(u)}{\partial x_j}
  &= q_k ~~~ k = 1,\ldots,d
\end{align}
where
\begin{equation}
	g_{jk} := \int^u{b_{jk}(s) ds}.
\label{eq:definition_g}
\end{equation}

Equations (\ref{eq:generalFirstOrderSystem_u}) and
(\ref{eq:generalFirstOrderSystem_q}) can now be treated by means of a standard
Discontinuous Galerkin scheme. The corresponding weak formulation for cell $K$ 
can be written as
\begin{align}
\label{eq:generalWeakFormulation_u}
	\int_K {\frac{\partial u^h}{\partial t} \varphi_u dx}
	- \int_K {(f_i(u^h) - b_{ij}q_j^h) \frac{\partial \varphi_u}{\partial x_i} dx}
	+ \int_{\partial K} {h_u(u^h, q^h, n, x, t) \varphi_u dS}
	&= \int_K {Q^h \varphi_u dx}\\
\label{eq:generalWeakFormulation_q}
	- \int_K {g_{ik} \frac{\partial \varphi_{q_k}}{\partial x_i} dx}
	+ \int_{\partial K} {h_{q_k}(u^h, q^h, n, x, t) \varphi_q dS}
	&= \int_K {q_k^h \varphi_{q_k} dx}
\end{align}
for $k = 1,\ldots,d$ (no summation over $k$!). Here, $u^h$, $q_k^h$ and $Q^h$
denote the Discontinuous Galerkin approximations of $u$, $q_k$ and $Q$,
$\varphi_u$ and $\varphi_{q_k}$ denote the test functions, $h_u$, $h_{q_k}$
denote the numerical fluxes and $n$ denotes the outward unit normal of cell
$K$.

For the formulation of the numerical fluxes, we use the definition of the 
average
\begin{equation}
	{p} = \frac{1}{2} (p^- + p^+)
\label{eq:averageOperator}
\end{equation}
and the jump
\begin{equation}
  [p] = p^+ - p^-
\label{eq:jumpOperator}
\end{equation}
of a scalar quantity $p$ at a border (where $p^-$ is the value inside the cell
and $p^+$ is the value on the other side of the interface). Now, the fluxes can
be formulated as
\begin{equation}
	h_u(u^h, q^h, n, x, t)
	= \hat{f}(u^{h-}, u^{h+}, n)
	- \frac{[g_{ij}]}{[u]} \{q_j\}n_i
	- c_i [q_i]
\label{eq:generalFlux_u}
\end{equation}
and
\begin{equation}
  h_{q_k}(u^h, q^h, n, x, t)
  = \{g_{ik}\}n_i
  - c_k[u]
\label{eq:generalFlux_q}
\end{equation}
where $c_i, \ldots, c_d$ are non-negative penalty coefficients and 
$\hat{f}$ is an arbitrary E-flux (see \cite{Osher1984}) consistent with the
nonlinearity
\begin{equation}
  f = f_i(u) n_i
\label{eq:nonlinearity}
\end{equation}

\section{One-dimensional Poisson equation}
\label{sec:poission1d}
Throughout this section we will consider the one-dimensional Poisson equation
\begin{equation}
	\Delta u(x) = 1 ~~~ x \in [0;1]
\label{eq:poisson}
\end{equation}
with the boundary conditions
\begin{equation}
	u(0) = u(1) = 0
\label{eq:poissonBoundaryCondition}
\end{equation}
and the exact solution
\begin{equation}
	u(x) = \frac{1}{2} (x^2 - x).
\label{eq:poissonExactSolution}
\end{equation}
In terms of the notation introduced in section \ref{sec:theoreticalBackground},
this leads to the following settings:
\begin{align}
  \frac{\partial u}{\partial t} &= 0\\
  d &= 1\\
  f(u) = f_1(u) &= 0\\
  \hat{f} &= -c_0 [u]\\
  a_{11}(u) = b_{11}(u) &= 1\\
  q = q_1 &= \frac{\partial u}{\partial x}\\
  Q &= -1\\
  g = g_{11} &= u.
\end{align}
Note that, even though the function $f$ itself is zero, $\hat{f}$ is not zero
but penalizes jumps of $u$ (with an additional, non-negative penalty parameter 
$c_0$) As a result, equations (\ref{eq:generalWeakFormulation_u}), 
(\ref{eq:generalWeakFormulation_q}), (\ref{eq:generalFlux_u}) and 
(\ref{eq:generalFlux_q}) can be reduced to
\begin{align}
  \label{equ:poisson_u_equation}
	- \int_K {(-q^h) \frac{\partial \varphi_u}{\partial x} dx}
	+ \int_{\partial K} {h_u(u^h, q^h, n, x, t) \varphi_u dS}
	&= \int_K {(-1) \varphi_u(x) dx}
	\\
  \label{equ:poisson_q_equation}
	- \int_K {u^h \frac{\partial \varphi_q}{\partial x} dx}
	+ \int_{\partial K} {h_{q}(u^h, q^h, n, x, t) \varphi_q dS}
	&= \int_K {q^h \varphi_{q} dx}
	\\
  \label{equ:poisson_u_flux}
	h_u(u^h, q^h, n, x, t)
	&= -c_0 [u]
	- \{q\}n_1
	- c_{1} [q]
	\\
  \label{equ:poisson_q_flux}
  h_{q}(u^h, q^h, n, x, t)
  &= \{u\}n_1
  - c_{1}[u].
\end{align}

\begin{lstlisting}[caption=BoSSS implementation of the left hand side of 
equation \ref{equ:poisson_u_equation} and of equation 
\ref{equ:poisson_u_flux}, label=lst:poisson_flux_u]
class uFlux : LinearFlux {
    
  [...]
    
  protected override double BorderEdgeFlux(ref InParams inp, double[] Uin) {
    return InnerEdgeFlux(ref inp, Uin, new double[] { 0.0, Uin[1] });
  }
      
  protected override double InnerEdgeFlux(ref InParams inp, double[] Uin, double[] Uout) {
    return c0 * (Uout[0] - Uin[0])
      - 0.5 * (Uin[1] + Uout[1]) * inp.normal[0]
      - c1 * (Uout[1] - Uin[1]);
  }

  protected override void Flux(double[] x, double[] parameters, double[] U, double[] output) {
    output[0] = -U[1];
  }
  
  public override IList<string> ArgumentOrdering {
    get { return new string[] { "u", "q" }; }
  }
}
\end{lstlisting}

This formulation can easily be transferred to the BoSSS syntax. The complete
program can be found in the sub-directory
\verb|src\public\L4-application\LDGPoisson1d| of any BoSSS developer
installation.

The implementation of equations \ref{equ:poisson_u_equation} and 
\ref{equ:poisson_u_flux} is shown in listing \ref{lst:poisson_flux_u}. The 
method \emph{InnerEdgeFlux} is a direct representation of equation
\ref{equ:poisson_u_flux} and needs no further explanation. The
\emph{BorderEdegeFlux} also corresponds to equation \ref{equ:poisson_u_flux},
but is only executed for interfaces which coincide with the boundary. Thus,
this method simply calls \emph{InnerEdgeFlux} while inserting the predefined
boundary value for $u^+$ (which is zero) in this case. Finally, the method
\emph{Flux} corresponds to the factor of the derivative of the test function
in the first integral of equation \ref{equ:poisson_u_equation}. In this simple
case, this factor is equal to $q^h$ which also is the return value of the
\emph{Flux} method in listing \ref{lst:poisson_flux_u}.

\begin{lstlisting}[caption=BoSSS implementation of the right hand side of
equation \ref{equ:poisson_u_equation},label=lst:poisson_source_u]
class uSource : LinearSource {
  
  [...]
  
  protected override double Source(double[] x, double[] parameters, double[] U) {
    return 1;
  }
  
  public override IList<string> ArgumentOrdering {
    get { return new string[0]; }
  }
}
\end{lstlisting}

The source term on the right hand side of equation \ref{equ:poisson_u_equation}
has not yet been considered. It is implemented as a separate class (see
\ref{lst:poisson_source_u}) and their coupling takes place later in the code. 
The syntax for the creation of such a source is straighforward but it should be
noticed that BoSSS expects an equation of the form $F = 0$ which is why the
method $Source$ returns $-Q$ (which is equal to $-1$ in our case).

\begin{lstlisting}[caption=BoSSS implementation of the left hand side of 
equation \ref{equ:poisson_q_equation} and of equation 
\ref{equ:poisson_q_flux}, label=lst:poisson_flux_q]
class qFlux : LinearFlux {
    
  [...]
    
  protected override double BorderEdgeFlux(ref InParams inp, double[] Uin) {
    return 0.0;
  }
      
  protected override double InnerEdgeFlux(ref InParams inp, double[] Uin, double[] Uout) {
    return 0.5 * (Uin[0] + Uout[0]) * inp.normal[0]
      - c1 * (Uout[0] - Uin[0]);
  }

  protected override void Flux(double[] x, double[] parameters, double[] U, double[] output) {
    output[0] = U[0];
  }
  
  public override IList<string> ArgumentOrdering {
    get { return new string[] { "u" }; }
  }
}
\end{lstlisting}

\begin{lstlisting}[caption=BoSSS implementation of the right hand side of
equation \ref{equ:poisson_q_equation}, label=lst:poisson_source_q]
class uSource : LinearSource {
  
  [...]
  
  protected override double Source(double[] x, double[] parameters, double[] U) {
    return -U[0];
  }
  
  public override IList<string> ArgumentOrdering {
    get { return new string[] { "q" }; }
  }
}
\end{lstlisting}

The listings \ref{lst:poisson_flux_q} and  \ref{lst:poisson_source_q} show an
analogous implementation of the equations \ref{equ:poisson_q_equation} and
\ref{equ:poisson_q_flux} with one major difference: Since $q$ is an auxiliary
variable there is no flux across boundary edges. Instead, the correct behaviour
of $q$ is solely enforced by the source $u$ which is why the method 
\emph{Flux} in listing \ref{lst:poisson_flux_q} return zero.

\begin{lstlisting}[caption=Assembly of the DG operator for the poisson 
equation, label=lst:poisson_operator]
class Program : BoSSS.Solution.Application {

  [...]
  
  protected override void CreateEquationsAndSolvers() {
    double c0 = 1.0;
    double c1 = 10.0;

    operator = new BoSSS.Foundation.SpatialDifferentialOperator(
    	new string[] { "u", "q" }, new string[] { "u", "q" });
    operator.EquationComponents["u"].Add(new uFlux(c0, c1));
    operator.EquationComponents["u"].Add(new uSource());
    operator.EquationComponents["q"].Add(new qFlux(c0));
    operator.EquationComponents["q"].Add(new qSource());
    operator.Commit();

    [...]
  }
}
\end{lstlisting}

In a last step, the created equations components have to be incorporated into
a system matrix. In BoSSS, this matrix can be calculated from a so called
\emph{SpatialDifferentialOperator} if it is configured with the components we
just created. Listing \ref{lst:poisson_operator} illustrates this procedure 
for our case. As one can see, this step finally connects the separately defined
source terms to the rest of the equation.
\section{Two-dimensional heat equation}
\label{sec:heat2d}
This chapter deals with the solution of the two-dimensional heat equation
\begin{equation}
  \frac{\partial u}{\partial t} - \kappa \left(
    \frac{\partial^2 u}{\partial x_1^2}
    + \frac{\partial^2 u}{\partial x_2^2}
  \right)
  ~~~ x = \binom{x_1}{x_2} \in \Omega = [0; 2\pi] \times [0; 2\pi]
\label{eq:heat}
\end{equation}
with the boundary conditions
\begin{equation}
  u_{\partial \Omega} = 0
\label{eq:heat_boundary}
\end{equation}
and the initial condition
\begin{equation}
  u_0 = sin(x_1) sin(x_2).
\label{eq:heat_initial}
\end{equation}

Using the framework described in section \ref{sec:theoreticalBackground}, this
setting can be expressed via
\begin{align}
  d &= 2\\
  \hat{f} &= -c_0 [u]\\
  b_{ij} &= \delta_{ij} \sqrt{\kappa}\\
  g_{ij} &= \delta_{ij} \sqrt{\kappa} u\\
  q_k &= \sqrt{\kappa} \frac{\partial u}{\partial x_k}\\
  Q &= 0
\end{align}
which leads to the formulation
\begin{align}
\label{eq:heat_u_equation}
	\int_K {\frac{\partial u^h}{\partial t} \varphi_u dx}
	- \int_K {
		-\sqrt{\kappa} q_1^h \frac{\partial \varphi_u}{\partial x_1}
		-\sqrt{\kappa} q_2^h \frac{\partial \varphi_u}{\partial x_2}
	dx}
	+ \int_{\partial K} {h_u(u^h, q^h, n, x, t) \varphi_u dS}
	&= 0\\
\label{eq:heat_q_equation}
	- \int_K {\sqrt{\kappa} u \frac{\partial \varphi_{q_k}}{\partial x_k} dx}
	+ \int_{\partial K} {h_{q_k}(u^h, q^h, n, x, t) \varphi_q dS}
	&= \int_K {q_k^h \varphi_{q_k} dx}
\end{align}
\begin{align}
\label{eq:heat_u_flux}
  h_u(u^h, q^h, n, x, t)
	&= -c_0 [u]
	- \sqrt{\kappa} (\{q_1\} n_1 + \{q_2\} n_2)
	- c_1 ([q_1] + [q_2])\\
\label{eq:heat_q_flux}
  h_{q_k}(u^h, q^h, n, x, t)
  &= \{u\} n_k
  - c_k[u]
\end{align}
where we have assumed $c_1=c_2$ for simplicity. This expression is equivalent
to the system
\begin{align}
\label{eq:heat_u_equation2}
	\int_K {\frac{\partial u^h}{\partial t} \varphi_u dx}
	- \int_K {
		-\kappa \tilde{q}_1^h \frac{\partial \varphi_u}{\partial x_1}
		-\kappa \tilde{q}_2^h \frac{\partial \varphi_u}{\partial x_2}
	dx}
	+ \int_{\partial K} {h_u(u^h, \tilde{q}^h, n, x, t) \varphi_u dS}
	&= 0\\
\label{eq:heat_q_equation2}
	- \int_K {u \frac{\partial \varphi_{q_k}}{\partial x_k} dx}
	+ \int_{\partial K} {h_{q_k}(u^h, \tilde{q}^h, n, x, t) \varphi_q dS}
	&= \int_K {\tilde{q}_k^h \varphi_{q_k} dx}
\end{align}
\begin{align}
\label{eq:heat_u_flux2}
  h_u(u^h, \tilde{q}^h, n, x, t)
	&= -c_0 [u]
	- \kappa (\{\tilde{q}_1\} n_1 + \{\tilde{q}_2\} n_2)
	- c_1 \sqrt{\kappa} ([\tilde{q}_1] + [\tilde{q}_2])\\
\label{eq:heat_q_flux2}
  h_{q_k}(u^h, \tilde{q}^h, n, x, t)
  &= \{u\} n_k
  - c_k[u]
\end{align}
where $q_k = \sqrt{\kappa} \tilde{q_k}$.

The full BoSSS implementation of the system
\ref{eq:heat_u_equation2}-\ref{eq:heat_q_flux2} is located in the 
sub-directory\linebreak \verb|src\public\L4-application\LDGHeat\| of any BoSSS developer 
installation. The listings provided in this section only highlight the most 
important differences compared to the treatment of a one-dimensional equation 
(like the Poisson equation discussed in section \ref{sec:poission1d}) in BoSSS.

\begin{lstlisting}[caption=BoSSS implementation of equations 
\ref{eq:heat_u_equation2} and \ref{eq:heat_u_flux2}, label=lst:heat_flux_u]
class uFlux : LinearFlux {
    
  [...]
    
  protected override double BorderEdgeFlux(ref InParams inp, double[] Uin) {
    return InnerEdgeFlux(ref inp, Uin, new double[] { 0.0, Uin[1], Uin[2] });
  }
  
  protected override void Flux(double[] x, double[] parameters, double[] U, double[] output) {
    output[0] = -m_kappa * U[1];
    output[1] = -m_kappa * U[2];
  }
  
  public override IList<string> ArgumentOrdering {
    get { return new string[] { "u", "q1", "q2" }; }
  }
}
\end{lstlisting}

Listing \ref{lst:heat_flux_u} shows the implementation of flux function for the
original variable $u$. Here, unlike the one-dimensional case, the \emph{Flux} 
function has to treat different fluxes associated with different spatial 
derivatives of the test function (cf. the second integral in equation
\ref{eq:heat_u_equation2}). That is why the array \emph{output} always has $d$ 
dimensions where \emph{output[i]} should represent the flux associated with the
spatial derivative of the test function in $x_i$ direction.

\begin{lstlisting}[caption=Assembly of the DG operator for the heat equation, 
label=lst:heat_operator]
class Program : BoSSS.Solution.Application {

  [...]
  
  protected override void CreateEquationsAndSolvers() {
    double c0 = 100.0;
    double c1 = 1.0;
    double kappa = 10.0;

    SpatialDifferentialOperator operator = new SpatialDifferentialOperator(
                3, 3, "u", "du/dx", "du/dy", "_1", "_2", "_3");
            operator.EquationComponents["_1"].Add(new Fluxes.uFlux(kappa, C0, C1));
            operator.EquationComponents["_2"].Add(new Fluxes.qFlux(C1, 0));
            operator.EquationComponents["_2"].Add(new Fluxes.LDGSource(0));
            operator.EquationComponents["_3"].Add(new Fluxes.qFlux(C1, 1));
            operator.EquationComponents["_3"].Add(new Fluxes.LDGSource(1));
            operator.Commit();

    [...]
  }
}
\end{lstlisting}

\begin{lstlisting}[caption=BoSSS implementation of the right hand side of
equation \ref{eq:heat_q_equation2}, label=lst:heat_source_q]
class uSource : LinearSource {
  
  [...]
  
  private string[] m_ArgumentMapping;
  
  public LDGSource(int d) {
    if (d == 0) {
      m_ArgumentMapping = new string[] { "du/dx" };
    } else if (d == 1) {
      m_ArgumentMapping = new string[] { "du/dy" };
    } else {
      throw new ArgumentException("Dimension must be 0 or 1");
    }
  }
  
  protected override double Source(double[] x, double[] parameters, double[] U) {
    return -U[0];
  }
  
  public override IList<string> ArgumentOrdering {
    get { return m_ArgumentMapping; }
  }
}
\end{lstlisting}

Having implemented the numerical fluxes, the 
\emph{SpatialDifferentialOperator} has to be assembled. This procedure is 
depicted in listing \ref{lst:heat_operator} where one can see that our operator
now consists of three equations (components \emph{\_1}, \emph{\_2} and 
\emph{\_3}) instead of two. Additionally, the constructors of the \emph{qFlux} 
and the \emph{LDGSource} consume an argument which indicates the direction of 
the associated derivative. An example for the usage of this parameter can be 
seen in listing \ref{lst:heat_source_q} the constructor uses the argument $d$
in order to determine if the derivative in $x_1$- ("`du/dx"') or 
$x_2$-direction ("`du/dy"') is required as a source term.


\bibliography{BibDatabase}

\end{document}
