
%
% FDY TEMPLATE for ANUAL REPORT
% ===================================================================
\NeedsTeXFormat{LaTeX2e}
\documentclass[11pt,twoside,a4paper]{fdyartcl}
\usepackage[utf8]{inputenc}
\usepackage[T1]{fontenc}
\usepackage[ngerman,english]{babel} % selectlanguage wird nur dann gebraucht, wenn
                                    % mehrere Sprachen-packages verwendet werden,
                                    % das zuletzt angegebene package ist die aktive
                                    % Sprache, mit \selectlanguage kann umgeschaltet
                                    % werden
\usepackage[intoc]{nomencl} % zur Erstellung einer Nomenklatur
                                   % Die Option [intoc] sorgt dafuer,
                                   % dass die Nomenklatur im
                                   % Inhaltsverzeichnis eingetragen
                                   % wird. Aufruf mit:
                                   % makeindex Diss.nlo -s nomencl.ist -o Diss.nls
\usepackage{longtable}  % fuer tabellen, die evtl ueber Seiten
                        % umgebrochen werden muessen
\usepackage{graphicx}   % Stellt \includegraphics zur Verfuegung
\usepackage{parskip}    % Setzt parindent auf null und parskip auf
                        % einen angemessenen Wert
\usepackage{calc}       % Erlaubt, verschiedene Masse zu addieren
                        % z.B 1cm+2pt
\usepackage[a4paper,twoside,outer=2.2cm,inner=3cm,top=1.5cm,bottom=2.7cm,includehead]{geometry}
                                        % erheblich verbesserte
                                        % Papieranassung
\usepackage{setspace}   % Stellt \singlespacing, \onehalfspacing und
                        % \doublespacig zur Verfuegung
                        % erlaubt ausserdem die Verwendung der
                        % Umgebung \begin{spacing}{2.3}
\setstretch{1.05}       % minimal vergroesserter Zeilenabstand
\usepackage{amsmath}    % Stellt verschiedene Mathematik Operatoren
                        % und Befehle bereit und verbessert die
                        % Darstellung von Gleichungen ermoeglicht
                        % ausserdem die Verwendung von \boldsymbol
                        % fuer z.B. griechische Buchstaben
\usepackage{amsfonts}
\usepackage{amsthm}
\usepackage{mathpazo}   % Aenderung der Standardschrift auf Palatino
\usepackage{upgreek}    % nicht-kursive grichische Buchstaben
\usepackage{fancyhdr}

\graphicspath{{./Figures/}}%
% \input hyphen_dt.tex  % Spezielle Trennregeln fuer deutsche
                        % Woerter - gehoert in den Vorspann
\clubpenalty = 10000%
\widowpenalty = 10000%
\displaywidowpenalty =10000


%%%%%%%%%%%%%%%%%%%%%%%%%%%%%%%%%%%%%%%%%%%%%%%%%%%%%%%%%%%%%%%%%%%%%
%%%%%%%%%%%%%%%%%%%%%%%%%%%%%%%%%%%%%%%%%%%%%%%%%%%%%%%%%%%%%%%%%%%%%
% TITLE AND AUTHOR
%%%%%%%%%%%%%%%%%%%%%%%%%%%%%%%%%%%%%%%%%%%%%%%%%%%%%%%%%%%%%%%%%%%%%
%%%%%%%%%%%%%%%%%%%%%%%%%%%%%%%%%%%%%%%%%%%%%%%%%%%%%%%%%%%%%%%%%%%%%
\title{How to use the CUBE profiling library in BoSSS\\TU Darmstadt - Chair of Fluid Dynamic (FDY)}
\author{Dennis Krause}
%%%%%%%%%%%%%%%%%%%%%%%%%%%%%%%%%%%%%%%%%%%%%%%%%%%%%%%%%%%%%%%%%%%%%
%%%%%%%%%%%%%%%%%%%%%%%%%%%%%%%%%%%%%%%%%%%%%%%%%%%%%%%%%%%%%%%%%%%%%
%%%%%%%%%%%%%%%%%%%%%%%%%%%%%%%%%%%%%%%%%%%%%%%%%%%%%%%%%%%%%%%%%%%%%

%%%%%%%%%%%%%%%%%%%%%%%%%%%%%%%%%%%%%%%%%%%%%%%%%%%%%%%%%%%%%%%%%%%%%
%%%%%%%%%%%%%%%%%%%%%%%%%%%%%%%%%%%%%%%%%%%%%%%%%%%%%%%%%%%%%%%%%%%%%
% Kopfzeile
\pagestyle{fancy}
\fancyhf{}
\fancyhead[EL]{\sffamily \small  \thepage\ }
\fancyhead[OR]{\sffamily \small  \thepage\ }
\fancyhead[OC]{\sffamily \small TU Darmstadt - FDY}
\fancyhead[EC]{\sffamily \small BoSSS tutorial}
%%%%%%%%%%%%%%%%%%%%%%%%%%%%%%%%%%%%%%%%%%%%%%%%%%%%%%%%%%%%%%%%%%%%%
%%%%%%%%%%%%%%%%%%%%%%%%%%%%%%%%%%%%%%%%%%%%%%%%%%%%%%%%%%%%%%%%%%%%%
%%%%%%%%%%%%%%%%%%%%%%%%%%%%%%%%%%%%%%%%%%%%%%%%%%%%%%%%%%%%%%%%%%%%%
%\renewcommand{\vec}[1]{\textrm{\textbf{#1}}}
\renewcommand{\vec}[1]{\mathbf{#1}}
\newcommand{\grvec}[1]{\boldsymbol{#1}}
\renewcommand{\div}[1]{\textrm{div}\left( #1 \right)}

\newcommand{\real}{\mathbb{R}}
\newcommand{\complex}{\mathbb{C}}
\newcommand{\realpos}{\mathbb{R}_{\geq0}}
\newcommand{\natWoZero}{\mathbb{N}_{>0}}
\newcommand{\natInclZero}{\mathbb{N}_{\geq0}}
\newcommand{\nat}{\mathbb{N}}
\newcommand{\integers}{\mathbb{Z}}
\newcommand{\code}[1]{\sffamily{#1}}
\newcommand{\coderm}[1]{\footnote{\code{#1}}}


\newcounter{cnt}

\newtheoremstyle{myPlain} % namei
 {15pt}% Space above
 {3pt}% Space below
 {\itshape}% Body font
 {}% Indent amount
 {\bfseries}% Theorem head font %\scshape
 {:}% Punctuation after theorem head
 {.5em}% hSpace after theorem headi2
 {\thmname{#1} \thmnumber{#2} \thmnote{ -- #3}}% Theorem head spec (can be left empty, meaning `normal')


\theoremstyle{myPlain}
%\theoremstyle{plain}
\newtheorem{myTheorem}[cnt]{Theorem}
%\newtheorem{satzDef}[cnt]{Satz und Definition}

\newtheoremstyle{myDefinition} % namei
 {15pt}% Space above
 {3pt}% Space below
 {}% Body font
 {}% Indent amount
 {\bfseries}% Theorem head font
 {:}% Punctuation after theorem head
 {.5em}% hSpace after theorem headi2
 {\thmname{#1} \thmnumber{#2} \thmnote{ -- #3}}% Theorem head spec (can be left empty, meaning `normal')

\theoremstyle{myDefinition}
%\theoremstyle{definition}
%\newtheorem{bsp}[cnt]{Beispiel}
\newtheorem{myDef}[cnt]{Definition}
\newtheorem{myRem}[cnt]{Remark}
\newtheorem{myNot}[cnt]{Notation}


%%%%%%%%%%%%%%%%%%%%%%%%%%%%%%%%%%%%%%%%%%%%%%%%%%%%%%%%%%%%%%%%%%%%%

%       DOKUMENT
\begin{document}

\pagenumbering{arabic} % Zurueckschalten auf arabische Ziffern, dabei wird der Zaehler auf 1 gesetzt


% Titelseite
% -------------------
\maketitle


\begin{abstract}
This document explains the installation of the CUBE library under Windows and its usage for BoSSS. This document refers to CUBE v4.3.4.
\end{abstract}


\textbf{The current guide is just a workaround until the CUBE library will be released for 64-bit Windows. After that, BoSSS is going to write its profiling data automatically in \textit{.cubex} files in your Session Directory.}

\section{Installation}
In the following the installation of CUBE on your local Windows workstation will be explained:
\begin{enumerate} 
	\item Download Cube win32 binary v4.3.4 from \\ \emph{http://www.scalasca.org/software/cube-4.x/download.html}.
	\item Start the installation by executing the \textit{.exe}.
	\item Copy the \textit{cubew.dll} from the installation folder \textit{\textbackslash CUBE\textbackslash bin} in your \\ \textit{\textbackslash BoSSS\textbackslash src\textbackslash experimental\textbackslash ilPSP.Cube\textbackslash bin\textbackslash Release } folder.	
	\item Open the \textit{ilPSP.Cube.csproj} with Visual Studio.
	\item Build ilPSP.Cube with the following options:
	\begin{itemize}
		\item Target Framework: .Net 4.5
		\item Platform target: x86 
	\end{itemize} 	
\end{enumerate} 
Now the \textit{ilPSP.Cube.exe} should appear in your Release folder.

\section{Usage}
\begin{enumerate}
	\item Copy your \textit{profiling\_bin.X.txt} files to \textit{\textbackslash BoSSS\textbackslash src\textbackslash experimental\textbackslash ilPSP.Cube\textbackslash bin\textbackslash Release}.
	\item Execute \textit{ilPSP.Cube.exe} and type number of processes (i.e. number of profiling files).
	\item After the conversion is finished you will find a new folder in your Release directory named after the following convention: \\  calc.p[\textit{number of processes}].r[\textit{number of repetition}]
	\item The folder contains all the profiling files and a new \textit{profile.cubex} file. You can open the \textit{.cubex} file by double click.
\end{enumerate}




\section{Useful functions}
If you discover some useful function using the CUBE GUI please insert a short description:

\begin{itemize}
	\item By choosing \textit{external percentage} you can relate your performance measurement to a reference measurement which will be 100 \%. 
\end{itemize}



\end{document} 