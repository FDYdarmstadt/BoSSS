\documentclass[BoSSSForSolvingConservationLaws.tex]{subfiles}
\begin{document}

For polynomial representation of a solution first we should choose the spatial dimension $D$ and order of representation $N$ to find $N_p$, number of basis polynomials.
\[
N_p=\frac{1}{D!}\prod_{1\le l \le D}(N+l)
\]
For our modal representation we use orthonormal basis polynomials on reference elements. Reference elements are defined such that their centroid is zero. For more information about the reference elements see \cite{KummerLayer2Manual09}. As an example here we show the second order basis polynomials defined in 2D on a square reference element defined as $[-1,1]^2$. For creating the desired basis polynomials first we choose set $V=\{1,x,y,x^2,xy,y^2\}$ and apply the Gram-Schmidt algorithm to make them orthogonal in the space of inner product $\langle f,g \rangle=\int_{K_{ref}} f(\mathbf{x})g(\mathbf{x})d\mathbf{x}$.
\begin{align*}
  &u_k=v_k-\sum_{j=1}^{k-1} \frac{\langle v_k,u_j \rangle}{\langle u_j,u_j \rangle}u_j, \qquad 1\leq k \leq 6\\
  &u_1=v_1=1, \qquad \|u_1\|=2\\
  &u_2=x-\frac{\langle x,1 \rangle}{\langle 1,1 \rangle}\cdot1=x, \qquad \|u_2\|=\frac{2}{\sqrt{3}}\\
  &\vdots
\end{align*}
Afterwards the polynomials are normalized dividing them by their L2-Norm defined as $\|f\|=\sqrt{\langle f,f \rangle}$. The resulting orthonormal polynomials on the defined square domain are
\[
\Phi=\bigg( \frac{1}{2},
    \frac{\sqrt{3}}{2}x,\frac{\sqrt{3}}{2}y,
    (\frac{3}{4}x^2-\frac{1}{4})\cdot\sqrt{5},\frac{3}{2}xy,(\frac{3}{4}y^2-\frac{1}{4})\cdot\sqrt{5}
  \bigg)
\]

\subsection*{Implementation in BoSSS}
Each member $\phi$ of a set $\Phi$ of orthonormal basis polynomials of order $N$, can be thought of as summation of some components $P$.
\[
P=C x^\alpha y^\beta z^\gamma \qquad \alpha, \beta, \gamma =0,\dots,l \qquad l \leq N
\]
A \emph{polynomial}\coderm{BoSSS.Foundation.Grid.Polynomial} $\phi$ is specified with an array of coefficients\coderm{BoSSS.Foundation.Grid.Polynomial.Coeff} and a D-dimensional array\coderm{BoSSS.Foundation.Grid.Polynomial.Exponents} of exponents. The \emph{orthonormal polynomials}\coderm{BoSSS.Foundation.Grid.Simplex.OrthonormalPolynomials} of a \emph{simplex}\coderm{BoSSS.Foundation.Grid.Simplex} (a reference element in BoSSS) are polynomials, constructed by \emph{add}ing\coderm{BoSSS.Foundation.Grid.Polynomial.AddCoeff} the components $P$ and are stored according to their degree in ascending order. Orthonormal polynomials up to certain degrees are available for each simplex. When a \emph{basis}\coderm{BoSSS.Foundation.Basis} is constructed to represent a field, \emph{polynomials}\coderm{BoSSS.Foundation.Basis.Polynomials} $\phi_n$ of the basis with the desired order are chosen\coderm{BoSSS.Foundation.Basis.Basis(...)} among the orthonormal polynomials defined for the corresponding grid simplex.

\end{document}
