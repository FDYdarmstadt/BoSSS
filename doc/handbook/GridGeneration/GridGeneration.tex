\BoSSSopen{GridGeneration/GridGeneration}
\graphicspath{{GridGeneration/GridGeneration.texbatch/}}

\BoSSScmd{
restart
 }
\BoSSSexeSilent
\BoSSScmd{
/// \section{Grid Generation}
///Firstly, we need to determine the boundaries of our grid/control volume. Is it important to know that the number of nodes needed are equal to the number of cells $+1$. For instance, for $10$ cells we need $11$ nodes. In this example we will use the Cartesian $2D$ grid from the database which requires $x$- and $y$-Nodes. The J term in the code is for doing a check if the desired resolution of the volume is correctly typed.
 }
\BoSSSexe
\BoSSScmd{
double[] xNodes = GenericBlas.Linspace(0, 1, Res + 1);\newline 
double[] yNodes = GenericBlas.Linspace(0, 1, Res + 1);\newline 
int J           = (xNodes.Length - 1)*(yNodes.Length - 1);\newline 
string GridName = string.Format(WorkflowMgm.CurrentProject + "\_J" +J);\newline 
 \newline 
Console.WriteLine("Creating grid with " + J + " cells. ");\newline 
 \newline 
GridCommons g;\newline 
g      = Grid2D.Cartesian2DGrid(xNodes, yNodes);\newline 
g.Name = GridName;
 }
\BoSSSexe
