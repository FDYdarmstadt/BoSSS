\documentclass[a4paper,10pt]{report} % book, amsbook, amsproc ?
\usepackage[utf8]{inputenc}
\usepackage[T1]{fontenc}
\usepackage[english]{babel}
\usepackage{calc}
\usepackage{amsmath}
\usepackage{amsfonts}
\usepackage{amssymb}
\usepackage{amsthm}
\usepackage[final]{graphicx} % Option "final" makes sure graphics also appear in draft mode
\usepackage[hang]{caption}
\usepackage[format=hang,justification=raggedright]{subfig}
\usepackage{booktabs}
\usepackage{siunitx}
\usepackage{algorithmicx}
\usepackage{algorithm}
\usepackage{algpseudocode}
\usepackage{pgfplots}
\usepackage{pstricks}
\usepackage{varwidth} % Minipage width variable size
\usepackage[centercolon=true]{mathtools} % \coloneqq
\usepackage{bm} % Bold greek smybols
\usepackage{eucal} % Nicer calligraphic symbols
\usepackage{upgreek}
\usepackage[nolist]{acronym}
\usepackage[left=2.00cm, right=2.00cm, top=2.00cm, bottom=2.00cm]{geometry}
\usepackage{url}
\usepackage{ifdraft}
\usepackage{import}
\usepackage{wsDBEtex}
\usepackage{color}
\usepackage{bibentry}
\usepackage{overpic}
\usepackage{tabularx}
\usepackage{float} % Make option [H] for figures work properly
\usepackage{varwidth}
\usepackage[toc,page]{appendix}
\usepackage[hidelinks]{hyperref}
%\usepackage[inline]{showlabels}
%\renewcommand{\showlabelfont}{\small\ttfamily\color{magenta}}
\usepackage{enumitem}
\usepackage{textcomp}

\usepackage[citestyle=authoryear,bibstyle=authoryear,backend=biber]{biblatex}
\usepackage{csquotes} % Recommended when using biblatex

% For external references
\addbibresource{BoSSShandbook.bib}

% Papers with the BoSSS tag on tubiblio
\addbibresource[label=BoSSSArticles,location=remote]{http://tubiblio.ulb.tu-darmstadt.de/cgi/search/archive/advanced/export_tubiblio_BibTeX.bib?screen=Search&dataset=archive&_action_export=1&output=BibTeX&exp=0|1|-date%2Fcreators_name%2Ftitle|archive|-|divisions%3Adivisions%3AANY%3AEQ%3Afb16_fdy|keywords%2Fkeywordsalternative_name%2Fkeywordsswd%3Akeywords%2Fkeywordsalternative_name%2Fkeywordsswd%3AALL%3AIN%3ABoSSS|type%3Atype%3AANY%3AEQ%3Aarticle|-|eprint_status%3Aeprint_status%3AANY%3AEQ%3Aarchive|metadata_visibility%3Ametadata_visibility%3AANY%3AEQ%3Ashow&n=&cache=1084536}

% PhD theses with the BoSSS tag on tubiblio
\addbibresource[label=BoSSSPhDTheses]{http://tubiblio.ulb.tu-darmstadt.de/cgi/search/archive/advanced/export_tubiblio_BibTeX.bib?dataset=archive&screen=Search&_action_export=1&output=BibTeX&exp=0|1|-date%2Fcreators_name%2Ftitle|archive|-|divisions%3Adivisions%3AANY%3AEQ%3Afb16_fdy|keywords%2Fkeywordsalternative_name%2Fkeywordsswd%3Akeywords%2Fkeywordsalternative_name%2Fkeywordsswd%3AALL%3AIN%3ABoSSS|type%3Atype%3AANY%3AEQ%3Athesis|-|eprint_status%3Aeprint_status%3AANY%3AEQ%3Aarchive|metadata_visibility%3Ametadata_visibility%3AANY%3AEQ%3Ashow&n=&cache=1084538}

% Master/Bachelor theses with the BoSSS tag on tubiblio
\addbibresource[label=BoSSSStudentTheses]{http://tubiblio.ulb.tu-darmstadt.de/cgi/search/archive/advanced/export_tubiblio_BibTeX.bib?screen=Search&dataset=archive&_action_export=1&output=BibTeX&exp=0|1|-date%2Fcreators_name%2Ftitle|archive|-|divisions%3Adivisions%3AANY%3AEQ%3Afb16_fdy|keywords%2Fkeywordsalternative_name%2Fkeywordsswd%3Akeywords%2Fkeywordsalternative_name%2Fkeywordsswd%3AALL%3AIN%3ABoSSS|type%3Atype%3AANY%3AEQ%3Amaster_thesis+bachelor_thesis+diplom_thesis|-|eprint_status%3Aeprint_status%3AANY%3AEQ%3Aarchive|metadata_visibility%3Ametadata_visibility%3AANY%3AEQ%3Ashow&n=&cache=1084540}


\usepackage{listings} % source code formatting, mainly for BoSSSPad reference
\lstset{
	language=[Sharp]C,
	basicstyle=\ttfamily\small\bfseries,
	commentstyle=\normalfont\slshape, %\color{greencomments},
	breaklines=true,
	frame=lines,
	showstringspaces=false
}

\setlength{\parindent}{0pt}
\setlength{\parskip}{5pt}

\pgfplotsset{
	compat=1.12,
	cycle list name=exotic
}

% General notation
\newcommand{\jump}[1]{\left[\!\left[{#1}\right]\!\right]}
\newcommand{\mean}[1]{\left\{{#1}\right\}}
\newcommand{\abs}[1]{\left\lvert{#1}\right\rvert}
\newcommand{\norm}[1]{\left\lVert{#1}\right\rVert}
\newcommand{\set}[1]{\mathcal{#1}}
%\renewcommand{\vec}[1]{\mathbf{#1}} % geht NICHT bei griechischen Buchstaben!
\renewcommand{\vec}[1]{\underline{#1}}
\renewcommand{\matrix}[1]{\underline{\underline{#1}}}
\newcommand{\diff}[1]{\,d\mkern-2mu#1}
\newcommand{\dV}{\diff{V}}
\newcommand{\dA}{\diff{A}}
\newcommand{\dt}{\diff{t}}
\newcommand{\Dt}{\Delta t}
\newcommand{\measure}[1]{\text{meas}({#1})}
\newcommand{\sign}[1]{\text{sgn}({#1})}
\newcommand{\del}[2]{\frac{\partial #1}{\partial #2}}
\newcommand{\restr}[2]{\left.#1\right|_{#2}}
\newcommand{\support}[1]{\text{supp}\left(#1\right)}
\newcommand{\LTwoNorm}[2]{\left\lVert#1\right\rVert_{L^2(#2)}}
\newcommand{\scp}[3][]{\left\langle #2 \middle| #3 \right\rangle_{#1}}

% Discretization
\newcommand{\domain}{\Omega}
\newcommand{\meshSize}{h}
\newcommand{\discrete}[1]{\tilde{#1}}
\newcommand{\discreteDomain}{\Omega{}_{\meshSize{}}}
\newcommand{\cell}{\set{K}}
\newcommand{\NumericalGrid}[1][h]{\mathfrak{K}_{#1}} 
\newcommand{\edge}{\set{E}}
\newcommand{\edges}{\mathfrak{E}_{h}} 
\newcommand{\normal}{\vec{n}}
\newcommand{\basis}{\Phi}
\newcommand{\degree}{P}
\newcommand{\flux}{f}
\newcommand{\fluxVec}{\vec{\flux}}
\newcommand{\diameter}[1]{\text{diam}({#1})}
\newcommand{\massMatrix}{\matrix{M}}
\newcommand{\real}{\mathbb{R}}
\newcommand{\poly}{\mathbb{P}}
\newcommand{\one}{\mathbf{1}}
\newcommand{\normm}{\vert\Vert}
\newcommand{\innen}{-}
\newcommand{\aussn}{+}
\newcommand{\sip}{\textrm{sip}}
\newcommand{\naive}{\textrm{naive}}
\newcommand{\basisRef}{\basis^{\text{ref}}}
\newcommand{\cellRef}{\cell^{\text{ref}}}
\newcommand{\Ns}{\text{Ns}}

% common abbreveations
\acrodef{dg}[DG]{Discontinuous Galerkin}
\acrodef{gui}[GUI]{graphical user interface}
\acrodef{ibm}[IBM]{immersed boundary method}
\acrodef{hpc}[HPC]{high performance computing}
\acrodef{ide}[IDE]{integrated development environment}
\acrodef{cns}[CNS]{compressible Navier-Stokes}


% Physical quantities (non dimensional)
\newcommand{\heatCapacityRatio}{\gamma}
\newcommand{\density}{\rho}
\newcommand{\velvec}{\vec{u}}
\newcommand{\momentum}{\density{}\mkern-1mu \velocity}
\newcommand{\momvec}{\vec{m}}
\newcommand{\energy}{\density{}\mkern-2mu E}
\newcommand{\pressure}{p}
\newcommand{\innerEnergy}{e}
\newcommand{\enthalpy}{\bar{h}}
\newcommand{\speedOfSound}{a}
\newcommand{\temperature}{T}
\newcommand{\velocity}{u}
\newcommand{\Reynolds}{\ensuremath{\mathrm{Re}}}
\newcommand{\reynolds}{\mathrm{Re}}
\newcommand{\Prandtl}{\ensuremath{\mathrm{Pr}}}
\newcommand{\Mach}{\ensuremath{\mathrm{Ma}}}
\newcommand{\Froude}{\ensuremath{\mathrm{Fr}}}
\newcommand{\stressTensor}{\tau}
\newcommand{\stress}{\matrix{\tau}}
\newcommand{\heatFlux}{q}
\newcommand{\heatSource}{Q}
\newcommand{\externalForces}{F}
\newcommand{\viscosity}{\mu}
\newcommand{\thermalConductivity}{k}
\newcommand{\gravitationalConstant}{g}
\newcommand{\signalVelocity}{S}
\newcommand{\aoa}{\alpha}
\newcommand{\lift}{c_L}
\newcommand{\liftForce}{F_L}
\newcommand{\drag}{c_D}
\newcommand{\dragForce}{F_D}
\newcommand{\entropy}{\bar{s}}

% Physical quantities (dimensional)
\newcommand{\dimensional}[1]{\tilde{#1}}
\newcommand{\reference}[1]{{#1}_{\infty}}




% Title Page
\usepackage{xcolor}
\usepackage{tikz}
\usepackage{subcaption} 
%\usepackage{graphicx}
\usepackage{amsmath}
\usepackage{amsfonts}
\usepackage{amsthm}
\usepackage{amssymb}
\usepackage{pgfplots}


\newcommand{\cut}{\textrm{X}}
\newcommand{\grid}{\mathfrak{K}}
\newcommand{\gridCC}{\mathfrak{K}_{\textrm{CC}}}
\newcommand{\gridNear}{\mathfrak{K}_{\textrm{near}}}
\newcommand{\gridNarrow}{\mathfrak{K}_{\textrm{narrow}}}
\newcommand{\gridSolid}{\mathfrak{K}_{\textrm{S}}}
\newcommand{\gridFluid}{\mathfrak{K}_{\textrm{F}}}
\newcommand{\gridCCM}{\grid^\cut}
\newcommand{\gridAgg}[1]{\grid^{\cut, \alpha, {#1}}}
\renewcommand{\vec}[1]{\textbf{#1}}

\newcommand{\spaceVk}{\mathbb{V}_{\vec{k}}^{\textrm{X},\alpha}}
\newcommand{\Pdg}{\mathbb{P}_k(\grid,t)}
\newcommand{\Pcg}{\mathbb{P}_{\tilde{k}}(\grid,t)}
\newcommand{\PolySpace}[2]{\mathbb{P}^{#1}_{#2}}
\newcommand{\CutPolySpace}[1]{\PolySpace{\cut}{#1}}
\newcommand{\AggCutPolySpace}[1]{\PolySpace{\cut,\alpha}{#1}}
\newcommand{\Matrix}[1]{\textbf{#1}}
\newcommand{\Vector}[1]{\textbf{#1}}
\newcommand{\Def}{\textbf{Definition}}
\newcommand{\Interface}{\mathcal{J}}
\renewcommand{\cupdot}{\dot{\cup}}
\newcommand{\Eqref}[1]{Equation \ref{#1}}
\newcommand{\Figref}[1]{Figure \ref{#1}}
\newcommand{\Tabref}[1]{Table \ref{#1}}
\newcommand{\GammaCC}{\Gamma_{CC}}
\newcommand{\GammaFluid}{\Gamma_{F}}
\newcommand{\GammaSolid}{\Gamma_{S}}
\newcommand{\BoSSS}{\textit{BoSSS}}


\begin{document}


\chapter{Parallel Performance and Scaling}
This section covers basic performance tests, i.e. how specific algorithms scale in parallel with increasing \emph{number of processors}. So far, all calculations for this research were conducted on the Lichtenberg high performance computer of the TU Darmstadt.

\section{ Xdg Poisson, steady droplet }
\subsection{weak scaling}
Weak scaling is investigated of the test problem introduced in \ref{sec:XdgPoisson}. This means the problem size per processor is constant during the study (in this case approximately 10.000 DOF / core). So the total problem size grows with increasing number of cores. The expectation is that wall clock time remains constant trough the study. This means in the same time frame with $N$ cores we are able to solve a system, which is $N$ times the size of a single core run.

In \ref{weakXdgPoissonScaling} the scaling of V-krylov-cycle is shown:

\graphicspath{{./apdx-MPISolverPerformance/weakScaling/XdgPoisson/plots/}} 
\begin{figure}[h!]
	\begin{center}
		% GNUPLOT: LaTeX picture with Postscript
\begingroup
  \makeatletter
  \providecommand\color[2][]{%
    \GenericError{(gnuplot) \space\space\space\@spaces}{%
      Package color not loaded in conjunction with
      terminal option `colourtext'%
    }{See the gnuplot documentation for explanation.%
    }{Either use 'blacktext' in gnuplot or load the package
      color.sty in LaTeX.}%
    \renewcommand\color[2][]{}%
  }%
  \providecommand\includegraphics[2][]{%
    \GenericError{(gnuplot) \space\space\space\@spaces}{%
      Package graphicx or graphics not loaded%
    }{See the gnuplot documentation for explanation.%
    }{The gnuplot epslatex terminal needs graphicx.sty or graphics.sty.}%
    \renewcommand\includegraphics[2][]{}%
  }%
  \providecommand\rotatebox[2]{#2}%
  \@ifundefined{ifGPcolor}{%
    \newif\ifGPcolor
    \GPcolortrue
  }{}%
  \@ifundefined{ifGPblacktext}{%
    \newif\ifGPblacktext
    \GPblacktexttrue
  }{}%
  % define a \g@addto@macro without @ in the name:
  \let\gplgaddtomacro\g@addto@macro
  % define empty templates for all commands taking text:
  \gdef\gplbacktext{}%
  \gdef\gplfronttext{}%
  \makeatother
  \ifGPblacktext
    % no textcolor at all
    \def\colorrgb#1{}%
    \def\colorgray#1{}%
  \else
    % gray or color?
    \ifGPcolor
      \def\colorrgb#1{\color[rgb]{#1}}%
      \def\colorgray#1{\color[gray]{#1}}%
      \expandafter\def\csname LTw\endcsname{\color{white}}%
      \expandafter\def\csname LTb\endcsname{\color{black}}%
      \expandafter\def\csname LTa\endcsname{\color{black}}%
      \expandafter\def\csname LT0\endcsname{\color[rgb]{1,0,0}}%
      \expandafter\def\csname LT1\endcsname{\color[rgb]{0,1,0}}%
      \expandafter\def\csname LT2\endcsname{\color[rgb]{0,0,1}}%
      \expandafter\def\csname LT3\endcsname{\color[rgb]{1,0,1}}%
      \expandafter\def\csname LT4\endcsname{\color[rgb]{0,1,1}}%
      \expandafter\def\csname LT5\endcsname{\color[rgb]{1,1,0}}%
      \expandafter\def\csname LT6\endcsname{\color[rgb]{0,0,0}}%
      \expandafter\def\csname LT7\endcsname{\color[rgb]{1,0.3,0}}%
      \expandafter\def\csname LT8\endcsname{\color[rgb]{0.5,0.5,0.5}}%
    \else
      % gray
      \def\colorrgb#1{\color{black}}%
      \def\colorgray#1{\color[gray]{#1}}%
      \expandafter\def\csname LTw\endcsname{\color{white}}%
      \expandafter\def\csname LTb\endcsname{\color{black}}%
      \expandafter\def\csname LTa\endcsname{\color{black}}%
      \expandafter\def\csname LT0\endcsname{\color{black}}%
      \expandafter\def\csname LT1\endcsname{\color{black}}%
      \expandafter\def\csname LT2\endcsname{\color{black}}%
      \expandafter\def\csname LT3\endcsname{\color{black}}%
      \expandafter\def\csname LT4\endcsname{\color{black}}%
      \expandafter\def\csname LT5\endcsname{\color{black}}%
      \expandafter\def\csname LT6\endcsname{\color{black}}%
      \expandafter\def\csname LT7\endcsname{\color{black}}%
      \expandafter\def\csname LT8\endcsname{\color{black}}%
    \fi
  \fi
    \setlength{\unitlength}{0.0500bp}%
    \ifx\gptboxheight\undefined%
      \newlength{\gptboxheight}%
      \newlength{\gptboxwidth}%
      \newsavebox{\gptboxtext}%
    \fi%
    \setlength{\fboxrule}{0.5pt}%
    \setlength{\fboxsep}{1pt}%
\begin{picture}(7920.00,6800.00)%
    \gplgaddtomacro\gplbacktext{%
      \csname LTb\endcsname%
      \put(717,3791){\makebox(0,0)[r]{\strut{}$10^{2}$}}%
      \csname LTb\endcsname%
      \put(717,5278){\makebox(0,0)[r]{\strut{}$10^{3}$}}%
      \csname LTb\endcsname%
      \put(717,6765){\makebox(0,0)[r]{\strut{}$10^{4}$}}%
      \csname LTb\endcsname%
      \put(2029,3541){\makebox(0,0){\strut{}$10^{1}$}}%
      \csname LTb\endcsname%
      \put(5070,3541){\makebox(0,0){\strut{}$10^{2}$}}%
      \csname LTb\endcsname%
      \put(7329,3791){\makebox(0,0)[l]{\strut{} }}%
      \csname LTb\endcsname%
      \put(7329,4163){\makebox(0,0)[l]{\strut{} }}%
      \csname LTb\endcsname%
      \put(7329,4535){\makebox(0,0)[l]{\strut{} }}%
      \csname LTb\endcsname%
      \put(7329,4906){\makebox(0,0)[l]{\strut{} }}%
      \csname LTb\endcsname%
      \put(7329,5278){\makebox(0,0)[l]{\strut{} }}%
      \csname LTb\endcsname%
      \put(7329,5650){\makebox(0,0)[l]{\strut{} }}%
      \csname LTb\endcsname%
      \put(7329,6022){\makebox(0,0)[l]{\strut{} }}%
      \csname LTb\endcsname%
      \put(7329,6393){\makebox(0,0)[l]{\strut{} }}%
      \csname LTb\endcsname%
      \put(7329,6765){\makebox(0,0)[l]{\strut{} }}%
    }%
    \gplgaddtomacro\gplfronttext{%
      \csname LTb\endcsname%
      \put(219,5278){\rotatebox{-270}{\makebox(0,0){\strut{}~wallclock time}}}%
      \csname LTb\endcsname%
      \put(3147,4280){\makebox(0,0)[l]{\strut{}Kcycle w. add.-Schwarz DG2}}%
      \csname LTb\endcsname%
      \put(3147,4001){\makebox(0,0)[l]{\strut{}2h limit}}%
    }%
    \gplgaddtomacro\gplbacktext{%
      \csname LTb\endcsname%
      \put(717,570){\makebox(0,0)[r]{\strut{}$10^{1}$}}%
      \csname LTb\endcsname%
      \put(717,3366){\makebox(0,0)[r]{\strut{}$10^{2}$}}%
      \csname LTb\endcsname%
      \put(2029,320){\makebox(0,0){\strut{}$10^{1}$}}%
      \csname LTb\endcsname%
      \put(5070,320){\makebox(0,0){\strut{}$10^{2}$}}%
      \csname LTb\endcsname%
      \put(7329,570){\makebox(0,0)[l]{\strut{} }}%
      \csname LTb\endcsname%
      \put(7329,850){\makebox(0,0)[l]{\strut{} }}%
      \csname LTb\endcsname%
      \put(7329,1129){\makebox(0,0)[l]{\strut{} }}%
      \csname LTb\endcsname%
      \put(7329,1409){\makebox(0,0)[l]{\strut{} }}%
      \csname LTb\endcsname%
      \put(7329,1688){\makebox(0,0)[l]{\strut{} }}%
      \csname LTb\endcsname%
      \put(7329,1968){\makebox(0,0)[l]{\strut{} }}%
      \csname LTb\endcsname%
      \put(7329,2248){\makebox(0,0)[l]{\strut{} }}%
      \csname LTb\endcsname%
      \put(7329,2527){\makebox(0,0)[l]{\strut{} }}%
      \csname LTb\endcsname%
      \put(7329,2807){\makebox(0,0)[l]{\strut{} }}%
      \csname LTb\endcsname%
      \put(7329,3086){\makebox(0,0)[l]{\strut{} }}%
      \csname LTb\endcsname%
      \put(7329,3366){\makebox(0,0)[l]{\strut{} }}%
      \csname LTb\endcsname%
      \put(819,3545){\makebox(0,0){\strut{} }}%
      \csname LTb\endcsname%
      \put(2101,3545){\makebox(0,0){\strut{} }}%
      \csname LTb\endcsname%
      \put(3382,3545){\makebox(0,0){\strut{} }}%
      \csname LTb\endcsname%
      \put(4664,3545){\makebox(0,0){\strut{} }}%
      \csname LTb\endcsname%
      \put(5945,3545){\makebox(0,0){\strut{} }}%
      \csname LTb\endcsname%
      \put(7227,3545){\makebox(0,0){\strut{} }}%
    }%
    \gplgaddtomacro\gplfronttext{%
      \csname LTb\endcsname%
      \put(219,1968){\rotatebox{-270}{\makebox(0,0){\strut{}iterations}}}%
      \csname LTb\endcsname%
      \put(4023,123){\makebox(0,0){\strut{}no of cores}}%
      \csname LTb\endcsname%
      \put(3147,780){\makebox(0,0)[l]{\strut{}Kcycle w. add.-Schwarz DG2}}%
    }%
    \gplbacktext
    \put(0,0){\includegraphics{Scaling_2}}%
    \gplfronttext
  \end{picture}%
\endgroup

	\end{center}
	\caption{
		Solver wall clock time vs. no of processors, for polynomial degree $k=2$ and approximately 10.000 DOF / processor, Grid partitioning with METIS (except 128 cores with predefined partitioning),
		for problem/Equation (\ref{eq:poisson-jump-problem-def}).
	}
	\label{fig:weakXdgPoissonScaling}
\end{figure}

\begin{figure}[h!]
	\begin{center}
		% GNUPLOT: LaTeX picture with Postscript
\begingroup
  \makeatletter
  \providecommand\color[2][]{%
    \GenericError{(gnuplot) \space\space\space\@spaces}{%
      Package color not loaded in conjunction with
      terminal option `colourtext'%
    }{See the gnuplot documentation for explanation.%
    }{Either use 'blacktext' in gnuplot or load the package
      color.sty in LaTeX.}%
    \renewcommand\color[2][]{}%
  }%
  \providecommand\includegraphics[2][]{%
    \GenericError{(gnuplot) \space\space\space\@spaces}{%
      Package graphicx or graphics not loaded%
    }{See the gnuplot documentation for explanation.%
    }{The gnuplot epslatex terminal needs graphicx.sty or graphics.sty.}%
    \renewcommand\includegraphics[2][]{}%
  }%
  \providecommand\rotatebox[2]{#2}%
  \@ifundefined{ifGPcolor}{%
    \newif\ifGPcolor
    \GPcolortrue
  }{}%
  \@ifundefined{ifGPblacktext}{%
    \newif\ifGPblacktext
    \GPblacktexttrue
  }{}%
  % define a \g@addto@macro without @ in the name:
  \let\gplgaddtomacro\g@addto@macro
  % define empty templates for all commands taking text:
  \gdef\gplbacktext{}%
  \gdef\gplfronttext{}%
  \makeatother
  \ifGPblacktext
    % no textcolor at all
    \def\colorrgb#1{}%
    \def\colorgray#1{}%
  \else
    % gray or color?
    \ifGPcolor
      \def\colorrgb#1{\color[rgb]{#1}}%
      \def\colorgray#1{\color[gray]{#1}}%
      \expandafter\def\csname LTw\endcsname{\color{white}}%
      \expandafter\def\csname LTb\endcsname{\color{black}}%
      \expandafter\def\csname LTa\endcsname{\color{black}}%
      \expandafter\def\csname LT0\endcsname{\color[rgb]{1,0,0}}%
      \expandafter\def\csname LT1\endcsname{\color[rgb]{0,1,0}}%
      \expandafter\def\csname LT2\endcsname{\color[rgb]{0,0,1}}%
      \expandafter\def\csname LT3\endcsname{\color[rgb]{1,0,1}}%
      \expandafter\def\csname LT4\endcsname{\color[rgb]{0,1,1}}%
      \expandafter\def\csname LT5\endcsname{\color[rgb]{1,1,0}}%
      \expandafter\def\csname LT6\endcsname{\color[rgb]{0,0,0}}%
      \expandafter\def\csname LT7\endcsname{\color[rgb]{1,0.3,0}}%
      \expandafter\def\csname LT8\endcsname{\color[rgb]{0.5,0.5,0.5}}%
    \else
      % gray
      \def\colorrgb#1{\color{black}}%
      \def\colorgray#1{\color[gray]{#1}}%
      \expandafter\def\csname LTw\endcsname{\color{white}}%
      \expandafter\def\csname LTb\endcsname{\color{black}}%
      \expandafter\def\csname LTa\endcsname{\color{black}}%
      \expandafter\def\csname LT0\endcsname{\color{black}}%
      \expandafter\def\csname LT1\endcsname{\color{black}}%
      \expandafter\def\csname LT2\endcsname{\color{black}}%
      \expandafter\def\csname LT3\endcsname{\color{black}}%
      \expandafter\def\csname LT4\endcsname{\color{black}}%
      \expandafter\def\csname LT5\endcsname{\color{black}}%
      \expandafter\def\csname LT6\endcsname{\color{black}}%
      \expandafter\def\csname LT7\endcsname{\color{black}}%
      \expandafter\def\csname LT8\endcsname{\color{black}}%
    \fi
  \fi
    \setlength{\unitlength}{0.0500bp}%
    \ifx\gptboxheight\undefined%
      \newlength{\gptboxheight}%
      \newlength{\gptboxwidth}%
      \newsavebox{\gptboxtext}%
    \fi%
    \setlength{\fboxrule}{0.5pt}%
    \setlength{\fboxsep}{1pt}%
\begin{picture}(7920.00,6800.00)%
    \gplgaddtomacro\gplbacktext{%
      \csname LTb\endcsname%
      \put(717,604){\makebox(0,0)[r]{\strut{}$10^{0}$}}%
      \csname LTb\endcsname%
      \put(717,2046){\makebox(0,0)[r]{\strut{}$10^{1}$}}%
      \csname LTb\endcsname%
      \put(717,3489){\makebox(0,0)[r]{\strut{}$10^{2}$}}%
      \csname LTb\endcsname%
      \put(717,4931){\makebox(0,0)[r]{\strut{}$10^{3}$}}%
      \csname LTb\endcsname%
      \put(717,6373){\makebox(0,0)[r]{\strut{}$10^{4}$}}%
      \csname LTb\endcsname%
      \put(2029,354){\makebox(0,0){\strut{}$10^{1}$}}%
      \csname LTb\endcsname%
      \put(5070,354){\makebox(0,0){\strut{}$10^{2}$}}%
      \csname LTb\endcsname%
      \put(7329,604){\makebox(0,0)[l]{\strut{} }}%
      \csname LTb\endcsname%
      \put(7329,1325){\makebox(0,0)[l]{\strut{} }}%
      \csname LTb\endcsname%
      \put(7329,2046){\makebox(0,0)[l]{\strut{} }}%
      \csname LTb\endcsname%
      \put(7329,2767){\makebox(0,0)[l]{\strut{} }}%
      \csname LTb\endcsname%
      \put(7329,3489){\makebox(0,0)[l]{\strut{} }}%
      \csname LTb\endcsname%
      \put(7329,4210){\makebox(0,0)[l]{\strut{} }}%
      \csname LTb\endcsname%
      \put(7329,4931){\makebox(0,0)[l]{\strut{} }}%
      \csname LTb\endcsname%
      \put(7329,5652){\makebox(0,0)[l]{\strut{} }}%
      \csname LTb\endcsname%
      \put(7329,6373){\makebox(0,0)[l]{\strut{} }}%
      \csname LTb\endcsname%
      \put(819,6552){\makebox(0,0){\strut{} }}%
      \csname LTb\endcsname%
      \put(2101,6552){\makebox(0,0){\strut{} }}%
      \csname LTb\endcsname%
      \put(3382,6552){\makebox(0,0){\strut{} }}%
      \csname LTb\endcsname%
      \put(4664,6552){\makebox(0,0){\strut{} }}%
      \csname LTb\endcsname%
      \put(5945,6552){\makebox(0,0){\strut{} }}%
      \csname LTb\endcsname%
      \put(7227,6552){\makebox(0,0){\strut{} }}%
    }%
    \gplgaddtomacro\gplfronttext{%
      \csname LTb\endcsname%
      \put(219,3488){\rotatebox{-270}{\makebox(0,0){\strut{}runtime of proc0}}}%
      \csname LTb\endcsname%
      \put(4023,157){\makebox(0,0){\strut{}no of cores}}%
      \csname LTb\endcsname%
      \put(5901,6162){\makebox(0,0)[l]{\strut{}Slv Iter}}%
      \csname LTb\endcsname%
      \put(5901,5883){\makebox(0,0)[l]{\strut{}Slv Init}}%
      \csname LTb\endcsname%
      \put(5901,5604){\makebox(0,0)[l]{\strut{}Agg Init}}%
      \csname LTb\endcsname%
      \put(5901,5325){\makebox(0,0)[l]{\strut{}Mtx ass}}%
    }%
    \gplbacktext
    \put(0,0){\includegraphics{Profiling_2}}%
    \gplfronttext
  \end{picture}%
\endgroup

	\end{center}
	\caption{
		profiling of the V-kcycle run for the same setting:
		for problem/Equation (\ref{eq:poisson-jump-problem-def}).
	}
	\label{fig:weakXdgPoisson-kcycle-profiling}
\end{figure}

\subsection{strong scaling}

%\graphicspath{{./apdx-MPISolverPerformance/strongScaling/XdgPoisson/plots/}} 
%\begin{figure}[h!]
%	\begin{center}
%		% GNUPLOT: LaTeX picture with Postscript
\begingroup
  \makeatletter
  \providecommand\color[2][]{%
    \GenericError{(gnuplot) \space\space\space\@spaces}{%
      Package color not loaded in conjunction with
      terminal option `colourtext'%
    }{See the gnuplot documentation for explanation.%
    }{Either use 'blacktext' in gnuplot or load the package
      color.sty in LaTeX.}%
    \renewcommand\color[2][]{}%
  }%
  \providecommand\includegraphics[2][]{%
    \GenericError{(gnuplot) \space\space\space\@spaces}{%
      Package graphicx or graphics not loaded%
    }{See the gnuplot documentation for explanation.%
    }{The gnuplot epslatex terminal needs graphicx.sty or graphics.sty.}%
    \renewcommand\includegraphics[2][]{}%
  }%
  \providecommand\rotatebox[2]{#2}%
  \@ifundefined{ifGPcolor}{%
    \newif\ifGPcolor
    \GPcolortrue
  }{}%
  \@ifundefined{ifGPblacktext}{%
    \newif\ifGPblacktext
    \GPblacktexttrue
  }{}%
  % define a \g@addto@macro without @ in the name:
  \let\gplgaddtomacro\g@addto@macro
  % define empty templates for all commands taking text:
  \gdef\gplbacktext{}%
  \gdef\gplfronttext{}%
  \makeatother
  \ifGPblacktext
    % no textcolor at all
    \def\colorrgb#1{}%
    \def\colorgray#1{}%
  \else
    % gray or color?
    \ifGPcolor
      \def\colorrgb#1{\color[rgb]{#1}}%
      \def\colorgray#1{\color[gray]{#1}}%
      \expandafter\def\csname LTw\endcsname{\color{white}}%
      \expandafter\def\csname LTb\endcsname{\color{black}}%
      \expandafter\def\csname LTa\endcsname{\color{black}}%
      \expandafter\def\csname LT0\endcsname{\color[rgb]{1,0,0}}%
      \expandafter\def\csname LT1\endcsname{\color[rgb]{0,1,0}}%
      \expandafter\def\csname LT2\endcsname{\color[rgb]{0,0,1}}%
      \expandafter\def\csname LT3\endcsname{\color[rgb]{1,0,1}}%
      \expandafter\def\csname LT4\endcsname{\color[rgb]{0,1,1}}%
      \expandafter\def\csname LT5\endcsname{\color[rgb]{1,1,0}}%
      \expandafter\def\csname LT6\endcsname{\color[rgb]{0,0,0}}%
      \expandafter\def\csname LT7\endcsname{\color[rgb]{1,0.3,0}}%
      \expandafter\def\csname LT8\endcsname{\color[rgb]{0.5,0.5,0.5}}%
    \else
      % gray
      \def\colorrgb#1{\color{black}}%
      \def\colorgray#1{\color[gray]{#1}}%
      \expandafter\def\csname LTw\endcsname{\color{white}}%
      \expandafter\def\csname LTb\endcsname{\color{black}}%
      \expandafter\def\csname LTa\endcsname{\color{black}}%
      \expandafter\def\csname LT0\endcsname{\color{black}}%
      \expandafter\def\csname LT1\endcsname{\color{black}}%
      \expandafter\def\csname LT2\endcsname{\color{black}}%
      \expandafter\def\csname LT3\endcsname{\color{black}}%
      \expandafter\def\csname LT4\endcsname{\color{black}}%
      \expandafter\def\csname LT5\endcsname{\color{black}}%
      \expandafter\def\csname LT6\endcsname{\color{black}}%
      \expandafter\def\csname LT7\endcsname{\color{black}}%
      \expandafter\def\csname LT8\endcsname{\color{black}}%
    \fi
  \fi
    \setlength{\unitlength}{0.0500bp}%
    \ifx\gptboxheight\undefined%
      \newlength{\gptboxheight}%
      \newlength{\gptboxwidth}%
      \newsavebox{\gptboxtext}%
    \fi%
    \setlength{\fboxrule}{0.5pt}%
    \setlength{\fboxsep}{1pt}%
\begin{picture}(7920.00,6800.00)%
    \gplgaddtomacro\gplbacktext{%
      \csname LTb\endcsname%
      \put(717,3791){\makebox(0,0)[r]{\strut{}$10^{2}$}}%
      \csname LTb\endcsname%
      \put(717,5278){\makebox(0,0)[r]{\strut{}$10^{3}$}}%
      \csname LTb\endcsname%
      \put(717,6765){\makebox(0,0)[r]{\strut{}$10^{4}$}}%
      \csname LTb\endcsname%
      \put(2029,3541){\makebox(0,0){\strut{}$10^{1}$}}%
      \csname LTb\endcsname%
      \put(5070,3541){\makebox(0,0){\strut{}$10^{2}$}}%
      \csname LTb\endcsname%
      \put(7329,3791){\makebox(0,0)[l]{\strut{} }}%
      \csname LTb\endcsname%
      \put(7329,4163){\makebox(0,0)[l]{\strut{} }}%
      \csname LTb\endcsname%
      \put(7329,4535){\makebox(0,0)[l]{\strut{} }}%
      \csname LTb\endcsname%
      \put(7329,4906){\makebox(0,0)[l]{\strut{} }}%
      \csname LTb\endcsname%
      \put(7329,5278){\makebox(0,0)[l]{\strut{} }}%
      \csname LTb\endcsname%
      \put(7329,5650){\makebox(0,0)[l]{\strut{} }}%
      \csname LTb\endcsname%
      \put(7329,6022){\makebox(0,0)[l]{\strut{} }}%
      \csname LTb\endcsname%
      \put(7329,6393){\makebox(0,0)[l]{\strut{} }}%
      \csname LTb\endcsname%
      \put(7329,6765){\makebox(0,0)[l]{\strut{} }}%
    }%
    \gplgaddtomacro\gplfronttext{%
      \csname LTb\endcsname%
      \put(219,5278){\rotatebox{-270}{\makebox(0,0){\strut{}~wallclock time}}}%
      \csname LTb\endcsname%
      \put(3147,4280){\makebox(0,0)[l]{\strut{}Kcycle w. add.-Schwarz DG2}}%
      \csname LTb\endcsname%
      \put(3147,4001){\makebox(0,0)[l]{\strut{}2h limit}}%
    }%
    \gplgaddtomacro\gplbacktext{%
      \csname LTb\endcsname%
      \put(717,570){\makebox(0,0)[r]{\strut{}$10^{1}$}}%
      \csname LTb\endcsname%
      \put(717,3366){\makebox(0,0)[r]{\strut{}$10^{2}$}}%
      \csname LTb\endcsname%
      \put(2029,320){\makebox(0,0){\strut{}$10^{1}$}}%
      \csname LTb\endcsname%
      \put(5070,320){\makebox(0,0){\strut{}$10^{2}$}}%
      \csname LTb\endcsname%
      \put(7329,570){\makebox(0,0)[l]{\strut{} }}%
      \csname LTb\endcsname%
      \put(7329,850){\makebox(0,0)[l]{\strut{} }}%
      \csname LTb\endcsname%
      \put(7329,1129){\makebox(0,0)[l]{\strut{} }}%
      \csname LTb\endcsname%
      \put(7329,1409){\makebox(0,0)[l]{\strut{} }}%
      \csname LTb\endcsname%
      \put(7329,1688){\makebox(0,0)[l]{\strut{} }}%
      \csname LTb\endcsname%
      \put(7329,1968){\makebox(0,0)[l]{\strut{} }}%
      \csname LTb\endcsname%
      \put(7329,2248){\makebox(0,0)[l]{\strut{} }}%
      \csname LTb\endcsname%
      \put(7329,2527){\makebox(0,0)[l]{\strut{} }}%
      \csname LTb\endcsname%
      \put(7329,2807){\makebox(0,0)[l]{\strut{} }}%
      \csname LTb\endcsname%
      \put(7329,3086){\makebox(0,0)[l]{\strut{} }}%
      \csname LTb\endcsname%
      \put(7329,3366){\makebox(0,0)[l]{\strut{} }}%
      \csname LTb\endcsname%
      \put(819,3545){\makebox(0,0){\strut{} }}%
      \csname LTb\endcsname%
      \put(2101,3545){\makebox(0,0){\strut{} }}%
      \csname LTb\endcsname%
      \put(3382,3545){\makebox(0,0){\strut{} }}%
      \csname LTb\endcsname%
      \put(4664,3545){\makebox(0,0){\strut{} }}%
      \csname LTb\endcsname%
      \put(5945,3545){\makebox(0,0){\strut{} }}%
      \csname LTb\endcsname%
      \put(7227,3545){\makebox(0,0){\strut{} }}%
    }%
    \gplgaddtomacro\gplfronttext{%
      \csname LTb\endcsname%
      \put(219,1968){\rotatebox{-270}{\makebox(0,0){\strut{}iterations}}}%
      \csname LTb\endcsname%
      \put(4023,123){\makebox(0,0){\strut{}no of cores}}%
      \csname LTb\endcsname%
      \put(3147,780){\makebox(0,0)[l]{\strut{}Kcycle w. add.-Schwarz DG2}}%
    }%
    \gplbacktext
    \put(0,0){\includegraphics{Scaling_2}}%
    \gplfronttext
  \end{picture}%
\endgroup

%	\end{center}
%	\caption{
%		Solver wall clock time vs. no of processors, for polynomial degree $k=2$ and approximately 10.000 DOF / processor, Grid partitioning with METIS (except 128 cores with predefined partitioning),
%		for problem/Equation (\ref{eq:poisson-jump-problem-def}).
%	}
%	\label{fig:weakXdgPoissonScaling}
%\end{figure}

%\begin{figure}[h!]
%	\begin{center}
%		% GNUPLOT: LaTeX picture with Postscript
\begingroup
  \makeatletter
  \providecommand\color[2][]{%
    \GenericError{(gnuplot) \space\space\space\@spaces}{%
      Package color not loaded in conjunction with
      terminal option `colourtext'%
    }{See the gnuplot documentation for explanation.%
    }{Either use 'blacktext' in gnuplot or load the package
      color.sty in LaTeX.}%
    \renewcommand\color[2][]{}%
  }%
  \providecommand\includegraphics[2][]{%
    \GenericError{(gnuplot) \space\space\space\@spaces}{%
      Package graphicx or graphics not loaded%
    }{See the gnuplot documentation for explanation.%
    }{The gnuplot epslatex terminal needs graphicx.sty or graphics.sty.}%
    \renewcommand\includegraphics[2][]{}%
  }%
  \providecommand\rotatebox[2]{#2}%
  \@ifundefined{ifGPcolor}{%
    \newif\ifGPcolor
    \GPcolortrue
  }{}%
  \@ifundefined{ifGPblacktext}{%
    \newif\ifGPblacktext
    \GPblacktexttrue
  }{}%
  % define a \g@addto@macro without @ in the name:
  \let\gplgaddtomacro\g@addto@macro
  % define empty templates for all commands taking text:
  \gdef\gplbacktext{}%
  \gdef\gplfronttext{}%
  \makeatother
  \ifGPblacktext
    % no textcolor at all
    \def\colorrgb#1{}%
    \def\colorgray#1{}%
  \else
    % gray or color?
    \ifGPcolor
      \def\colorrgb#1{\color[rgb]{#1}}%
      \def\colorgray#1{\color[gray]{#1}}%
      \expandafter\def\csname LTw\endcsname{\color{white}}%
      \expandafter\def\csname LTb\endcsname{\color{black}}%
      \expandafter\def\csname LTa\endcsname{\color{black}}%
      \expandafter\def\csname LT0\endcsname{\color[rgb]{1,0,0}}%
      \expandafter\def\csname LT1\endcsname{\color[rgb]{0,1,0}}%
      \expandafter\def\csname LT2\endcsname{\color[rgb]{0,0,1}}%
      \expandafter\def\csname LT3\endcsname{\color[rgb]{1,0,1}}%
      \expandafter\def\csname LT4\endcsname{\color[rgb]{0,1,1}}%
      \expandafter\def\csname LT5\endcsname{\color[rgb]{1,1,0}}%
      \expandafter\def\csname LT6\endcsname{\color[rgb]{0,0,0}}%
      \expandafter\def\csname LT7\endcsname{\color[rgb]{1,0.3,0}}%
      \expandafter\def\csname LT8\endcsname{\color[rgb]{0.5,0.5,0.5}}%
    \else
      % gray
      \def\colorrgb#1{\color{black}}%
      \def\colorgray#1{\color[gray]{#1}}%
      \expandafter\def\csname LTw\endcsname{\color{white}}%
      \expandafter\def\csname LTb\endcsname{\color{black}}%
      \expandafter\def\csname LTa\endcsname{\color{black}}%
      \expandafter\def\csname LT0\endcsname{\color{black}}%
      \expandafter\def\csname LT1\endcsname{\color{black}}%
      \expandafter\def\csname LT2\endcsname{\color{black}}%
      \expandafter\def\csname LT3\endcsname{\color{black}}%
      \expandafter\def\csname LT4\endcsname{\color{black}}%
      \expandafter\def\csname LT5\endcsname{\color{black}}%
      \expandafter\def\csname LT6\endcsname{\color{black}}%
      \expandafter\def\csname LT7\endcsname{\color{black}}%
      \expandafter\def\csname LT8\endcsname{\color{black}}%
    \fi
  \fi
    \setlength{\unitlength}{0.0500bp}%
    \ifx\gptboxheight\undefined%
      \newlength{\gptboxheight}%
      \newlength{\gptboxwidth}%
      \newsavebox{\gptboxtext}%
    \fi%
    \setlength{\fboxrule}{0.5pt}%
    \setlength{\fboxsep}{1pt}%
\begin{picture}(7920.00,6800.00)%
    \gplgaddtomacro\gplbacktext{%
      \csname LTb\endcsname%
      \put(717,604){\makebox(0,0)[r]{\strut{}$10^{0}$}}%
      \csname LTb\endcsname%
      \put(717,2046){\makebox(0,0)[r]{\strut{}$10^{1}$}}%
      \csname LTb\endcsname%
      \put(717,3489){\makebox(0,0)[r]{\strut{}$10^{2}$}}%
      \csname LTb\endcsname%
      \put(717,4931){\makebox(0,0)[r]{\strut{}$10^{3}$}}%
      \csname LTb\endcsname%
      \put(717,6373){\makebox(0,0)[r]{\strut{}$10^{4}$}}%
      \csname LTb\endcsname%
      \put(2029,354){\makebox(0,0){\strut{}$10^{1}$}}%
      \csname LTb\endcsname%
      \put(5070,354){\makebox(0,0){\strut{}$10^{2}$}}%
      \csname LTb\endcsname%
      \put(7329,604){\makebox(0,0)[l]{\strut{} }}%
      \csname LTb\endcsname%
      \put(7329,1325){\makebox(0,0)[l]{\strut{} }}%
      \csname LTb\endcsname%
      \put(7329,2046){\makebox(0,0)[l]{\strut{} }}%
      \csname LTb\endcsname%
      \put(7329,2767){\makebox(0,0)[l]{\strut{} }}%
      \csname LTb\endcsname%
      \put(7329,3489){\makebox(0,0)[l]{\strut{} }}%
      \csname LTb\endcsname%
      \put(7329,4210){\makebox(0,0)[l]{\strut{} }}%
      \csname LTb\endcsname%
      \put(7329,4931){\makebox(0,0)[l]{\strut{} }}%
      \csname LTb\endcsname%
      \put(7329,5652){\makebox(0,0)[l]{\strut{} }}%
      \csname LTb\endcsname%
      \put(7329,6373){\makebox(0,0)[l]{\strut{} }}%
      \csname LTb\endcsname%
      \put(819,6552){\makebox(0,0){\strut{} }}%
      \csname LTb\endcsname%
      \put(2101,6552){\makebox(0,0){\strut{} }}%
      \csname LTb\endcsname%
      \put(3382,6552){\makebox(0,0){\strut{} }}%
      \csname LTb\endcsname%
      \put(4664,6552){\makebox(0,0){\strut{} }}%
      \csname LTb\endcsname%
      \put(5945,6552){\makebox(0,0){\strut{} }}%
      \csname LTb\endcsname%
      \put(7227,6552){\makebox(0,0){\strut{} }}%
    }%
    \gplgaddtomacro\gplfronttext{%
      \csname LTb\endcsname%
      \put(219,3488){\rotatebox{-270}{\makebox(0,0){\strut{}runtime of proc0}}}%
      \csname LTb\endcsname%
      \put(4023,157){\makebox(0,0){\strut{}no of cores}}%
      \csname LTb\endcsname%
      \put(5901,6162){\makebox(0,0)[l]{\strut{}Slv Iter}}%
      \csname LTb\endcsname%
      \put(5901,5883){\makebox(0,0)[l]{\strut{}Slv Init}}%
      \csname LTb\endcsname%
      \put(5901,5604){\makebox(0,0)[l]{\strut{}Agg Init}}%
      \csname LTb\endcsname%
      \put(5901,5325){\makebox(0,0)[l]{\strut{}Mtx ass}}%
    }%
    \gplbacktext
    \put(0,0){\includegraphics{Profiling_2}}%
    \gplfronttext
  \end{picture}%
\endgroup

%	\end{center}
%	\caption{
%		profiling of the V-kcycle run for the same setting:
%		for problem/Equation (\ref{eq:poisson-jump-problem-def}).
%	}
%	\label{fig:weakXdgPoisson-kcycle-profiling}
%\end{figure}


\section{Stokes rotating sphere}
In this section we will stick to a rigid body $\Omega_S$ which is embedded in the computational domain $\Omega \subset \mathbb{R}^D$, and the fluid domain is given as 
$\Omega_F(t)=\Omega \setminus \overline{\Omega}_S(t)$.
We further assume the incompressible Navier-Stokes equation:



\begin{equation}
\left\{ \begin{array} {rclll}
\frac{\partial \rho \vec{u}}{\partial t}+ \nabla \cdot ( \rho \vec{u} \otimes \vec{u}) + \nabla p - \eta \Delta \vec{u} & = & \vec{f} \quad & in \ \ \Omega_F(t) \times (0,T) \\
\nabla \cdot \vec{u} & = & 0 \quad & in \ \ \Omega_F(t) \times (0,T) \\
\vec{u}(\vec{x},0) & = & \vec{0} \quad & on \ \ \Omega_F(0) \\
p(\vec{x},0) & = & 0 \quad & on \ \ \Omega_F(0) \\
\vec{u}(\vec{x},t) & = & \vec{0} \quad & on \ \ \partial\Omega_F \setminus \Interface \\
\vec{u}(\vec{x},t) & = & \boldsymbol{\omega}(t) \times \vec{r} \quad & on \ \ \Interface = \partial\Omega_S \cap \partial\Omega_F 
\end{array} \right.
\end{equation}
where rigid body $\Omega_S(t)$ is rotating with the angular velocity of $\boldsymbol{\omega}$. At the domain boundary $\partial\Omega_F \setminus \Interface$ and at the interface dirichlet boundary conditions are imposed. The surface $\partial \Omega_S$, respectively $\Interface$, is represented by the isocontour of the level-set $\varphi(x,t)$.

A sphere is defined by the isocontour of: 
\begin{equation}
	-x_1^2-x_2^2-x_3^2+r^2=0.
\end{equation}

\subsection{parallel efficiency (weak scaling)}
\begin{figure}[H]
	\centering
	\pgfplotsset{xlabel = cores, ylabel= Efficiency $T(8)/T(P)$, width = 0.6\textwidth, height= 0.45\textwidth}
	\begin{tikzpicture}
	\begin{loglogaxis}[
	xmin=8 ,xmax=256,
	legend columns=-1,
	%legend to name=bastingComparisonLegend,
	ymin=0,ymax=1,
	xtick={8,16,32,64,128,256},
	xticklabels={$\mathsf{8}$, $\mathsf{16}$, $\mathsf{32}$, $\mathsf{64}$, $\mathsf{128}$, $\mathsf{256}$}, 
	legend style={at={(.5,1.1)},anchor=north},
	]
	\def \WeakPath {./apdx-MPISolverPerformance/weakScaling/DG_rotSphere/plots}
	
	\addplot[mark=none, red] table[y=Speedup] {\WeakPath/weak_k2.dat};
	\addplot[mark=none, blue] table[y=Speedup] {\WeakPath/weak_k3.dat};
	\addplot[mark=none, orange] table[y=Speedup] {\WeakPath/weak_k4.dat};
	\legend{ $k2$, $k3$, $k4$ };
	\end{loglogaxis}
	\end{tikzpicture}
	
	\caption{weak scaling efficiency, weak scaling: constant degrees of freedom per core, sweep from 8 to 256 cores }
	\label{plt:weak_scaling}
\end{figure}

\subsection{parallel speedup (strong scaling)}
\begin{figure}[H]
	\centering
	\pgfplotsset{xlabel = cores, ylabel= Speedup $T(8)/T(P)$, width = 0.6\textwidth, height= 0.45\textwidth}
	\begin{tikzpicture}
	\begin{loglogaxis}[
	xmin=8 ,xmax=256,
	legend columns=-1,
	%legend to name=bastingComparisonLegend,
	ymin=1,ymax=32,
	xtick={8,16,32,64,128,256},
	xticklabels={$\mathsf{8}$, $\mathsf{16}$, $\mathsf{32}$, $\mathsf{64}$, $\mathsf{128}$, $\mathsf{256}$}, 
	legend style={at={(.5,1.1)},anchor=north},
	]
	\def \StrongPath {./apdx-MPISolverPerformance/strongScaling/DG_rotSphere/plots}
	
	\addplot[mark=none, red] table[y=Speedup] {\StrongPath/strong_k2.dat};
	\addplot[mark=none, blue] table[y=Speedup] {\StrongPath/strong_k3.dat};
	\addplot[mark=none, orange] table[y=Speedup] {\StrongPath/strong_k4.dat};
	\legend{ $k2$, $k3$, $k4$ };
	\end{loglogaxis}
	\end{tikzpicture}
	
	\caption{maximum runtime per linear solver iteration over all cores, sweep from 8 to 256 cores, }
	\label{plt:strong_scaling}
\end{figure}

\subsection{profile}
\begin{figure}[H]
	\centering
	\pgfplotsset{xlabel = cores, ylabel= runtime per iteration [sec] , width = 0.6\textwidth, height= 0.45\textwidth}
	\begin{tikzpicture}
	\begin{axis}
	[ybar stacked, enlargelimits=0.25,
	symbolic x coords={8, 16, 32, 64, 128, 256},
	%nodes near coords,
	%nodes near coords xbar stacked configuration/.style={},
	xtick=data,
	%every node near coord/.append style={xshift=5pt},
	legend style={at={(1.1,.5)},anchor=west},
	%totals/.style={nodes near coords align={anchor=south}},
	%x tick label style={anchor=south,yshift=-0.5cm},
	]
	%MatrixAssembly_time	AggregationBaseInit_time	DataIO_time	CGProjection_time	SayeCompile_time	StandardCompile_time	AMR_time	LoadBal_time	SolverInit_time	SolverRun_time
	
	\def \ProfilePath {./apdx-MPISolverPerformance/strongScaling/DG_rotSphere/plots/}
	
	
	\addplot table[y=AggregationBaseInit_time] {\ProfilePath};
	\addplot table[y=DataIO_time] {\ProfilePath};
	%\addplot table[y=CGProjection_time] {\ProfilePath};
	\addplot table[y=SayeCompile_time] {\ProfilePath};
	\addplot table[y=StandardCompile_time] {\ProfilePath};
	%\addplot table[y=AMR_time] {\ProfilePath};
	\addplot table[y=LoadBal_time] {\ProfilePath};
	\addplot table[y=SolverInit_time] {\ProfilePath};
	\addplot table[y=MatrixAssembly_time] {\ProfilePath};
	\addplot table[y=SolverRun_time,nodes near coords, nodes near coords align={horizontal}] {\ProfilePath};
	
	%\legend{ $MatrixAssembly_time$, $AggregationBaseInit_time$, $DataIO_time$, $CGProjection_time$, $SayeCompile_time$, $StandardCompile_time$, $AMR_time$, $LoadBal_time$, $SolverInit_time$, $SolverRun_time$ };
	\legend{  $AggregationBaseInit_time$, $DataIO_time$, $SayeCompile_time$, $StandardCompile_time$, $LoadBal_time$, $SolverInit_time$, $MatrixAssembly_time$, $SolverRun_time$ };
	\end{axis}
	\end{tikzpicture}
	\caption{k4, runtime profile of IBM solver}
	\label{plt:profiling}
\end{figure}

\end{document}